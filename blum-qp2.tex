\documentclass[,doc,floatsintext]{apa6}
\usepackage{lmodern}
\usepackage{amssymb,amsmath}
\usepackage{ifxetex,ifluatex}
\usepackage{fixltx2e} % provides \textsubscript
\ifnum 0\ifxetex 1\fi\ifluatex 1\fi=0 % if pdftex
  \usepackage[T1]{fontenc}
  \usepackage[utf8]{inputenc}
\else % if luatex or xelatex
  \ifxetex
    \usepackage{mathspec}
  \else
    \usepackage{fontspec}
  \fi
  \defaultfontfeatures{Ligatures=TeX,Scale=MatchLowercase}
\fi
% use upquote if available, for straight quotes in verbatim environments
\IfFileExists{upquote.sty}{\usepackage{upquote}}{}
% use microtype if available
\IfFileExists{microtype.sty}{%
\usepackage{microtype}
\UseMicrotypeSet[protrusion]{basicmath} % disable protrusion for tt fonts
}{}
\usepackage{hyperref}
\hypersetup{unicode=true,
            pdftitle={On the locality of vowel harmony over multi-tiered autosegmental representations},
            pdfauthor={Eileen Blum},
            pdfkeywords={keywords},
            pdfborder={0 0 0},
            breaklinks=true}
\urlstyle{same}  % don't use monospace font for urls
\usepackage{graphicx,grffile}
\makeatletter
\def\maxwidth{\ifdim\Gin@nat@width>\linewidth\linewidth\else\Gin@nat@width\fi}
\def\maxheight{\ifdim\Gin@nat@height>\textheight\textheight\else\Gin@nat@height\fi}
\makeatother
% Scale images if necessary, so that they will not overflow the page
% margins by default, and it is still possible to overwrite the defaults
% using explicit options in \includegraphics[width, height, ...]{}
\setkeys{Gin}{width=\maxwidth,height=\maxheight,keepaspectratio}
\IfFileExists{parskip.sty}{%
\usepackage{parskip}
}{% else
\setlength{\parindent}{0pt}
\setlength{\parskip}{6pt plus 2pt minus 1pt}
}
\setlength{\emergencystretch}{3em}  % prevent overfull lines
\providecommand{\tightlist}{%
  \setlength{\itemsep}{0pt}\setlength{\parskip}{0pt}}
\setcounter{secnumdepth}{5}
% Redefines (sub)paragraphs to behave more like sections
\ifx\paragraph\undefined\else
\let\oldparagraph\paragraph
\renewcommand{\paragraph}[1]{\oldparagraph{#1}\mbox{}}
\fi
\ifx\subparagraph\undefined\else
\let\oldsubparagraph\subparagraph
\renewcommand{\subparagraph}[1]{\oldsubparagraph{#1}\mbox{}}
\fi

%%% Use protect on footnotes to avoid problems with footnotes in titles
\let\rmarkdownfootnote\footnote%
\def\footnote{\protect\rmarkdownfootnote}

%%% Change title format to be more compact
\usepackage{titling}

% Create subtitle command for use in maketitle
\newcommand{\subtitle}[1]{
  \posttitle{
    \begin{center}\large#1\end{center}
    }
}

\setlength{\droptitle}{-2em}
  \title{On the locality of vowel harmony over multi-tiered autosegmental
representations}
  \pretitle{\vspace{\droptitle}\centering\huge}
  \posttitle{\par}
  \author{Eileen Blum\textsuperscript{1}}
  \preauthor{\centering\large\emph}
  \postauthor{\par}
  \date{}
  \predate{}\postdate{}

\shorttitle{Locality of vh over multi-tiered ARs}
\authornote{

Correspondence concerning this article should be addressed to Eileen Blum, 18 Seminary Place, New Brunswick, NJ 08901. E-mail: eileen.blum@rutgers.edu}
\affiliation{
\vspace{0.5cm}
\textsuperscript{1} Rutgers University}
\keywords{keywords\newline\indent Word count: X}
\usepackage{csquotes}
\usepackage{upgreek}
\captionsetup{font=singlespacing,justification=justified}

\usepackage{longtable}
\usepackage{lscape}
\usepackage{multirow}
\usepackage{tabularx}
\usepackage[flushleft]{threeparttable}
\usepackage{threeparttablex}

\newenvironment{lltable}{\begin{landscape}\begin{center}\begin{ThreePartTable}}{\end{ThreePartTable}\end{center}\end{landscape}}

\makeatletter
\newcommand\LastLTentrywidth{1em}
\newlength\longtablewidth
\setlength{\longtablewidth}{1in}
\newcommand{\getlongtablewidth}{\begingroup \ifcsname LT@\roman{LT@tables}\endcsname \global\longtablewidth=0pt \renewcommand{\LT@entry}[2]{\global\advance\longtablewidth by ##2\relax\gdef\LastLTentrywidth{##2}}\@nameuse{LT@\roman{LT@tables}} \fi \endgroup}


\usepackage{lineno}

\linenumbers
\usepackage{tipa}
\usepackage{gb4e}
\noautomath
\usepackage{tikz}
\usetikzlibrary{matrix}
\tikzset{marked/.style={draw=none, fill=none}}
\usepackage{mathptmx}
\usepackage{moresize}
\setlength{\parindent}{2em}

\usepackage{amsthm}
\newtheorem{theorem}{Theorem}[section]
\newtheorem{lemma}{Lemma}[section]
\theoremstyle{definition}
\newtheorem{definition}{Definition}[section]
\newtheorem{corollary}{Corollary}[section]
\newtheorem{proposition}{Proposition}[section]
\theoremstyle{definition}
\newtheorem{example}{Example}[section]
\theoremstyle{definition}
\newtheorem{exercise}{Exercise}[section]
\theoremstyle{remark}
\newtheorem*{remark}{Remark}
\newtheorem*{solution}{Solution}
\begin{document}
\maketitle

\section{Introduction}\label{introduction}

This qualifying paper aims to investigate the locality of vowel harmony
patterns using forbidden substructure constraints (FSCs) over
multi-tiered autosegmental representations (ARs). The investigation
provides a well-defined, computationally motivated theory of
well-formedness in vowel harmony. Jardine (2017) developed a theory of
tonal well-formedness and determined that tone patterns are
fundamentally local over two-tiered ARs. Investigating the locality of
vowel harmony patterns allows for a theory of well-formedness that makes
accurate typological predictions.

This qualifying paper analyzes vowel harmony as a phonotactic
restriction using only surface ARs rather than an input-output map.
Previous work has analyzed vowel harmony patterns as resulting from a
single assimilation process, whether it be feature spreading or
agreement (Bakovic, 2000; Clements, 1976; McCarthy, 2011; Nevins, 2010;
Rose \& Walker, 2011; van der Hulst \& Smith, 1986; Walker, 2010).
However, this paper shows that given a uniform theory of surface
markedness constraints, vowel harmony patterns utilize surface ARs that
can reflect either type of assimilation. The paper further argues that
vowel harmony is strictly local on the surface because all such patterns
can be captured by FSCs, which refer to the successor relation between
features and/or their associations to the vowels in a word.

First, the locality of a traditional spreading pattern will be
demonstrated for Akan. The basic spreading pattern with blocking vowels
in Akan is captured by a single FSC, which forbids an AR with two
different features on one tier and only a single feature on another
tier. In this way, Akan demonstrates how different feature tiers can
interact to restrict the possible ARs of vowel harmony.

In addition, the underspecification of features has been used to account
for vowel harmony patterns with transparent vowels. Feature
underspecification means that a language does not associate some vowels
to any feature on a particular tier. In such cases, the spread of a
feature on that tier is able to skip over the unassociated vowels. This
qualifying paper, however, shows that transparent vowels are associated
to features on all the same tiers as features that harmonizing vowels
are associated to. Rather than being underspecified, a vowel is
transparent based on the specific features it associates to. Thus,
transparent vowels in Finnish are captured by FSCs without relying on
language-specific underspecification.

While underspecification is shown to be unnecessary for vowels, this
paper will show that some languages do underspecify boundaries such that
they are represented on the segmental tier, but not on feature tiers.
For example, this paper will show that morpheme boundaries must be
represented on both the segmental and feature tiers in order for FSCs to
capture Turkish suffix harmony with disharmonic roots. On the other
hand, the Finnish root and suffix harmony patterns do not require
morpheme boundaries to be represented on feature tiers. The distinction
between the representations in these two languages will be argued to
result from the type of graph primitives the language utilizes for the
derivation of its ARs.

Lastly, this qualifying paper will show that FSCs over multi-tiered ARs
can also capture an unattested sour grapes vowel harmony pattern.
Regardless of whether word boundaries are specified on feature tiers or
not, their presence in an AR allows the theory outlined here to
represent an unattested pattern. Sour grapes is a vowel harmony pattern
in which a single blocking vowel prevents the spread of a feature
regardless of how many vowels intervene. Despite being unattested in the
literature, the current theory predicts such a pattern is possible. So,
further work remains to determine whether or not the multi-tiered ARs
posited here are too powerful to represent only attested vowel harmony
patterns.

A goal shared by all of generative phonology is to distinguish attested
patterns from logically possible, but unattested ones. A theory of
well-formedness in vowel harmony that accomplishes this goal must be
both expressive enough to explain the attested typology of vowel harmony
patterns and restrictive enough to exclude the logically possible
unattested vowel harmony patterns. While this qualifying paper does not
accomplish the goal of distinguishing attested from unattested vowel
harmony patterns, it adopts a formal language theory approach that
provides explicit ways of determining the locality of vowel harmony
patterns. This approach can then be used in the future to investigate
whether the current surface well-formedness theory can be restricted
further such that unattested vowel harmony patterns are not captured by
FSCs.

\subsection{The formal language theory
approach}\label{the-formal-language-theory-approach}

The goal of distinguishing attested phonological patterns from possible
unattested patterns is currently being investigated using formal
language theory to determine the expressive power required to compute
phonological patterns in general. The Chomsky hierarchy, in
(\ref{chomsky.hierarchy}), classifies stringsets in terms of the
relative expressivity of the grammars needed to generate them. Each
class that is lower on the hierarchy is also a proper subset of the
class above it.

\begin{exe}
\ex{The Chomsky Hierarchy:}\label{chomsky.hierarchy} \\ 
Finite $\subsetneq$ Regular $\subsetneq$ Context-Free $\subsetneq$ Context-Sensitive $\subsetneq$ Computably Enumerable
\end{exe}

A significant body of work in computational phonology shows that
phonological generalizations are properly contained within the regular
class of stringsets (Heinz \& Idsardi, 2013). Recent work has further
established a subregular hierarchy of stringset classes, i.e.~star-free
(SF) and weaker classes (Heinz, Rawal, \& Tanner, 2011; Rogers \&
Pullum, 2011; Rogers et al., 2013). A generative phonlogical theory must
be expressive enough to predict the regular patterns and restrictive
enough to rule out patterns that fall into a larger class, such as
context-free. The classifications of stringsets and ARs in this manner
are not directly comparable, but Jardine (2018; following Jardine and
Heinz, 2015a) provides a method for comparing the expressivity of the
grammars that generate them. Rogers et al. (2013) provides a cognitive
interpretation of string well-formedness whereby the well-formedness of
a string can be checked by scanning that string with a window of size k
to ensure that it does not contain the forbidden substructure of size k.
Jardine (2018) thus establishes a sub-SF class of \enquote{forbidden
k-factor grammars} over ARs, ASL\textsuperscript{g}, that is expressive
enough to capture a range of attested tone patterns
(ASL\textsuperscript{g\textsubscript{T}}). The goal of this qualifying
paper is to determine the suitability of multi-tiered ARs for capturing
vowel harmony patterns using forbidden k-factor grammars. Future work
will then be able to compare sets of multi-tiered ARs to existing
subregular grammars in order to classify vowel harmony patterns with
respect to the subregular hierarchy.

Patterns represented with multi-tiered ARs demonstrate whether or not
enriching the representation necessarily increases the expressivity of a
grammar. Representations of vowel harmony refer to subsegmental
features, which will be represented using multiple featural tiers, such
that each feature occupies a separate tier that is associated to a vowel
on the segmental tier (following Clements, 1976; McCarthy, 1988). Such
ARs include at least one additional tier compared with the ARs of tone
patterns, which utilize only two tiers (Jardine, 2016, 2017, 2018). This
qualifying paper will determine whether or not multi-tiered ARs
adequately capture vowel harmony patterns so that their expressivity can
eventually be compared to two-tiered ARs of tone. Three aspects of
multi-tiered ARs of vowel harmony are investigated: the complexity of
vowel harmony patterns with neutral vowels in Akan and Finnish,
generalizations that include domain information in Turkish, and whether
or not multi-tiered ARs predict the generation of an unattested pattern:
\enquote{sour grapes} (McCarthy, 2011; Padgett, 1995; Walker, 2010).
Each of these investigations will provide additional evidence for the
suitability of forbidden k-factor grammars over multi-tiered ARs for
generating vowel harmony patterns.

Vowel harmony can be viewed either as an input-output map or as a
phonotactic \enquote{cooccurrence restriction upon the vowels that may
occur in a word} (Clements, 1976). Some previous analyses use ARs to
describe vowel harmony patterns as the spreading of a vowel feature from
one vowel throughout the word until it is blocked (Clements, 1976;
Goldsmith, 1976; McCarthy, 1988; Padgett, 2002; Sagey, 1986; van der
Hulst, 2017; Walker, 2010, 2014). Clements (1976)'s well-formedness
condition motivates feature spreading in order to ensure that all
elements on one tier of an AR are connected via an association relation
to some element on another tier of the same AR. The result is an AR in
which all elements on one tier are associated to some element on another
tier. Many scholars have thus viewed vowel harmony as mapping an input
with a vowel feature associated to one vowel onto an output where that
same feature is associated to multiple vowels. However, a hierarchy that
classifies sets of ARs-- based on the Chomsky and related subregular
hierarchies --differs significantly from a parallel hierarchy for sets
of pairs of ARs, such as in a transformation (or map) from underlying to
surface form. Some influential work has been dedicated to classifying
input-output maps in phonology (i.e.~from underlying to surface form) as
stricly local within a Chomsky-based hierarchy of sets of pairs of
strings and demonstrating their learnability (Chandlee \& Heinz, 2018;
Chandlee \& Jardine, 2013; Chandlee, Eyraud, \& Heinz, 2014).

However, in this paper vowel harmony is viewed as a phonotactic
restriction such that only surface ARs are analyzed. The analysis of
surface representations utilizes a different understanding of
terminology than what is traditionally assumed for vowel harmony. For
example, vowel harmony is considered an assimilation pattern, which
means that vowels in a word are associated to the same feature. On the
surface, vowels can be associated to the same feature in one of two
ways: spreading or agreement. In this paper, a spreading AR is one in
which multiple vowels are associated to a single feature; thus spreading
is equated with multiple association. An agreement AR is one in which
vowels are associated to different iterations of the same feature with
the same value because there is a feature that intervenes with a
different value on the same tier.

This qualifying paper constitutes the first formal language theoretic
study of vowel harmony as a phonotactic restriction rather than an
input-output map. It will be taking a slightly different approach than
has been taken before by evaluating only the restrictions on output
substructures. While vowel harmony has been considered a derivational
process, this paper aims to determine the locality of only the surface
restrictions on vowel harmony patterns over multi-tiered ARs. The
harmonizing ARs that will be examined contain at least one feature that
is associated to more than one vowel, as it would be on the surface.
Ignoring input structures in this way allows for the eventual
classification of vowel harmony within the sub-SF hierarchy of patterns,
which in turn allows for the comparison of vowel harmony with other
phonological patterns that have been classified on the same hierarchy,
such as tone in Jardine (2018).

The structure of the remainder of this paper is as follows: Section 2 is
devoted to the representations with discussion of the motivations in 2.1
and assumptions in 2.2 that are adopted throughout the paper as well as
a definition of FSCs in 2.3 and an explanation of different assimilation
processes 2.4. Section 3 includes the analysis of two languages that
exemplify vowel harmony patterns with neutral--- i.e.~blocking and
transparent --- vowels. Section 4 analyzes a domain restricted vowel
harmony pattern in Turkish. Section 5 discusses how the system laid out
so far captures the unattested sour grapes pattern. And section 6
concludes.

\section{Defining Autosegmental Representations
(ARs)}\label{defining-autosegmental-representations-ars}

This qualifying paper will determine the locality of surface
restrictions on vowel harmony patterns over multi-tiered ARs by
investigating whether they can be captured using Jardine (2017)'s
\enquote{forbidden substructure constraints}(FSCs). This section
outlines the motivations for adopting the representations used
throughout this paper and the basic assumptions and definitions needed
to use them for analysis.

\subsection{Multi-tiered ARs}\label{multi-tiered-ars}

Autosegmental representations (ARs) of tonal patterns generally consist
of two tiers: the TBU and segmental tiers (Goldsmith, 1976; Jardine,
2016, 2017), but an open question that remains is: From a formal
perspective, what is the range of patterns that can be represented using
more than two autosegmental tiers? This paper investigates the
expressive power needed to represent one such set of patterns. Vowel
harmony patterns refer to subsegmental features, which will be
represented using multiple featural tiers; each feature occupies a
separate tier that is associated to a vowel on the segmental tier
(following Clements, 1976; McCarthy, 1988). For example, assuming binary
features, vowel features like {[}\(\pm\) back{]}, {[}\(\pm\) high{]},
etc. are represented on separate tiers and associated to a vowel on the
segmental tier, as in (\ref{fsc.ex}). Association relations are
represented by straight lines that connect elements (segments and
features) on different tiers. Where a tier consists of multiple
elements, the successor ordering relation between elements on that tier
is represented by arrows.

\begin{exe}
\ex \label{fsc.ex}
  \begin{tikzpicture}[baseline=(current bounding box.north)]
  \matrix [matrix of nodes,
          row sep=1em, 
          nodes={text height=1em, text depth=0.5em}] {
|(h)| $\pm$ high\\
|(v)| V         \\
|(b)| $\pm$ back\\
  };
   \draw (h.south) -- (v.north);
   \draw (v.south) -- (b.north);
  \end{tikzpicture}
\end{exe}

The goal for this project is to extend the work of Jardine (2017) to
determine whether vowel harmony patterns are local over ARs with more
than two tiers, as in (\ref{fsc.ex}). This qualifying paper evaluates
whether or not the restrictions on attested vowel harmony patterns can
be captured using FSCs that contain elements of more than one feature
tier.

\subsection{Representational
assumptions}\label{representational-assumptions}

Use of ARs requires discussion of at least some of the basic
representational assumptions held throughout this paper. The basic
assumptions are taken from Clements (1976)'s Well-Formedness Condition,
which includes stipulations of \emph{Full Specification} (FS), the
\emph{No Crossing Constraint} (NCC) (Goldsmith, 1976; Sagey, 1986), and
the \emph{Obligatory Contour Principle} (OCP) (Leben, 1973). Examples of
structures that violate each of these assumptions are shown in
(\ref{akan.ex2})-(\ref{akan.ex4}) below.

\begin{exe}
\ex \label{akan.ex2} Violates FS
  \begin{tikzpicture}[baseline=(current bounding box.north)]
  \matrix [matrix of nodes, row sep=2.5ex, column sep=2.25ex, nodes={text height=1em, text depth=0.5em}] 
  {
* & |(b)| -ATR & \\
  & |(d)| V    & |(e)| V \\
  & |(f)| -low & \\
  };
  \draw (b.south) -- (d.north);
  \draw (d.south) -- (f.north);
  \draw[thick,black,->] (d) -- (e);
  \draw foreach \x in {d, e} {(\x.south) -- (f.north)};
  \end{tikzpicture}
\end{exe}

\begin{exe}
\ex \label{akan.ex3} Violates NCC
  \begin{tikzpicture}[baseline=(current bounding box.north)]
  \matrix [matrix of nodes, row sep=2.5ex, column sep=2.25ex, nodes={text height=1em, text depth=0.5em}] 
  {
* & |(a)| +ATR & |(b)| -ATR \\
  & |(c)| V    & |(d)| V \\
  & |(e)| -low &  \\
  };
  \draw (a.south) -- (d.north);
  \draw (b.south) -- (c.north);
  \draw[thick,black,->] (a) -- (b);
  \draw[thick,black,->] (c) -- (d);
  \draw foreach \x in {c, d} {(\x.south) -- (e.north)};
  \end{tikzpicture}
\end{exe}

\begin{exe}
\ex \label{akan.ex4} Violates OCP
  \begin{tikzpicture}[baseline=(current bounding box.north)]
  \matrix [matrix of nodes, row sep=2.5ex, column sep=2.25ex, nodes={text height=1em, text depth=0.5em}] 
  {
* & |(a)| -ATR & |(b)| -ATR \\
  & |(c)| V    & |(d)| V \\
  & |(e)| -low & |(f)| -low \\
  };
  \draw (a.south) -- (c.north);
  \draw (b.south) -- (d.north);
  \draw (c.south) -- (e.north);
  \draw (d.south) -- (f.north);
  \draw[thick,black,->] (a) -- (b);
  \draw[thick,black,->] (c) -- (d);
  \draw[thick,black,->] (e) -- (f);
  \end{tikzpicture}
\end{exe}

First, FS means that each featural element must be associated to at
least one vowel on the segmental tier and each vowel on the segmental
tier must be associated to at least one element on each featural tier.
FS crucially allows vowels to be associated to multiple featural tiers
as is necessary for each vowel feature to occupy its own tier. The
hypothetical representation in (\ref{akan.ex2}) straighforwardly
violates FS because there is a vowel that is not associated to any
feature on the ATR tier. While both vowels are associated to a single
-low feature, the second vowel is not associated to any feature on the
ATR tier. Since vowel harmony patterns will be analyzed, it will be
assumed that consonants cannot be associated to vowel features and that
FS and vowel harmony in general ignore consonantal elements on the
segmental tier.

Second, the NCC states that association lines between the segmental tier
and a feature tier never cross. Odden (1994) adds that the NCC can only
evaluate the association between the segmental and one featural tier at
a time. The representation in (\ref{akan.ex3}) violates the NCC because
+ATR precedes -ATR, but is associated to a vowel that is preceded by a
vowel associated to -ATR; this configuration creates visually crossed
association lines.

A notable effect of FS along with the NCC is that they prevent what have
been called gapped structures (Archangeli \& Pulleyblank, 1994; Ringen
\& Vago, 1998). A gapped structure is one in which a feature appears to
have skipped over a vowel that it could potentially be associated to. FS
would prevent gapped structures in which the \enquote{skipped} vowel is
not associated to anything on that particular feature's tier. The NCC
would prevent gapped structures in which the surrounding two vowels are
associated to the same feature and the \enquote{skipped} vowel is
associated to a different feature on the same tier.

Lastly, the OCP stipulates that successive featural elements must be
distinct. The representation in (\ref{akan.ex4}) violates the OCP
because on both the ATR and low feature tiers there are two identical
successive features, -ATR and -low respectively. The OCP in conjunction
with FS results in representations where multiple vowels are associated
to a single feature rather than having multiple successive iterations of
the same feature each associated to a single vowel. An example
representation of an Akan word that satisfies all of the AR properties
discussed here is shown in (\ref{akan.ex}).

Both the NCC and the OCP have also been derived via a concatenation
operation (\(\circ\)) that merges autosegmental \enquote{graph
primitives}(Jardine \& Heinz, 2015a, p. 1). An autosegmental graph
primitive consists of an element on the segmental tier, the elements on
each feature tier and the associations between the featural and
segmental tiers. The concatenation operation combines a finite set of
adjacent graph primitives to generate a fully specified AR. For example,
the AR in (\ref{akan.ex}) is derived from the set of graph primitives in
(\ref{concat.ex}). Each primitive in (\ref{concat.ex}) is concatenated
with a single adjacent primitive. If two adjacent primitives share an
identical feature those two features are merged into one feature with
two associations, as in (\ref{akan.ex}). The merging of identical
adjacent features essentially prevents surface ARs from having multiple
iterations of a feature and crossed associations, thus satisfying both
the OCP and the NCC. However, if two segmental elements are associated
to the exact same feature and a different element intervenes then both
iterations of that feature will occur in the surface AR because only
adjacent primitive elements are concatenated and can thus be merged.
This qualifying paper will show that an intervening element can be a
vowel associated to the same feature with a different value or a domain
boundary. It will further show that a domain boundary primitive may
include that boundary on both segmental and feature tiers.

\begin{exe}
\ex \label{concat.ex} Concatenation of adjacent autosegmental graph primitives \\
  \begin{tikzpicture}[baseline=(current bounding box.north)]
  \matrix [matrix of nodes, row sep=2.5ex, column sep=2.25ex, nodes={text height=1em, text depth=0.5em}] 
  {
  &                                     & |(a)| -ATR &                                     & |(b)| -ATR \\
t & \node {}; \draw(0, 0) circle (3pt); & |(1)| i    & \node {}; \draw(0, 0) circle (3pt); & |(2)| e \\
  &                                     & |(c)| -low &                                     & |(d)| -low \\
  };
  \draw (a.south) -- (1.north);
  \draw (1.south) -- (c.north);
  \draw (b.south) -- (2.north);
  \draw (2.south) -- (d.north);
  \end{tikzpicture}
\end{exe}

\begin{exe}
\ex \label{akan.ex} Satisfies FS, NCC, and OCP \\
  \begin{tikzpicture}[baseline=(current bounding box.north)]
  \matrix [matrix of nodes, row sep=2.5ex, column sep=2.25ex, nodes={text height=1em, text depth=0.5em}] 
  {
  &            & |(b)| -ATR & \\
  & |(c)| t    & |(d)| i    & |(e)| e \\
  &            & |(f)| -low & \\
  };
  \draw foreach \x in {d, e} {(b.south) -- (\x.north)};
  \draw foreach \x in {d, e} {(\x.south) -- (f.north)};
  \draw[thick,black,->] (c) -- (d);
  \draw[thick,black,->] (d) -- (e);
  \end{tikzpicture}
\end{exe}

Again, the initial consonant in (\ref{akan.ex}) cannot be associated to
a vowel feature. While it is ordered with respect to the vowels, FS does
not require the consonant to be associated to any element on either
feature tier. The AR of \emph{tie} satisfies FS because each vowel is
associated to a feature on each of the featural tiers and all features
are associated to at least one vowel. The AR of \emph{tie} also
satisfies both the NCC and the OCP because there is only one of each
feature. The features are represented on separate tiers so association
lines cannot cross and there is nothing else on those tiers that could
violate the OCP. In addition, (\ref{akan.ex2})-(\ref{akan.ex})
illustrate that, unlike the usual notation, this paper will be adding a
representation of the successor ordering relation on each tier using
arrows.

\subsection{Definition of Constraints}\label{definition-of-constraints}

As mentioned above, this qualifying paper will use Jardine (2017)'s
\enquote{forbidden substructure constraints} to determine the locality
of surface restrictions on vowel harmony patterns over multi-tiered ARs.
Previous work on the logical descriptions of formal languages and their
applications to phonological well-formedness constraints (Heinz et al.,
2011; Rogers et al., 2013) led to the development of the theory of a
forbidden substructure grammar (following Jardine, 2017). A forbidden
substructure grammar is a logical statement of the form in (\ref{fsg})
below. Such a grammar will generate a set of well-formed structures that
does not contain any of r\(_1\) through r\(_n\).

\begin{exe}
\ex Forbidden substructure grammar (Jardine, 2017) \label{fsg} \\
$\neg$r$_1$ $\wedge$ $\neg$r$_2$ $\wedge$ $\neg$r$_3$ $\wedge$ ... $\wedge$ $\neg$r$_n$
\end{exe}

Negative well-formedness constraints are not new to phonological theory,
however. Optimality Theory (OT; Prince \& Smolensky, 1993, 2004)
introduced surface markedness constraints, which evaluate the
well-formedness of potential output structures (Jardine \& Heinz,
2015b). deLacy (2011) then called for \enquote{constraint definition
languages} in order to explicitly define the possible range of such
constraints and their interpretations. Jardine (2016) and Jardine (2017)
introduced the forbidden substructure grammars, which refer to
phonological structures and are both restrictive and computationally
local. The logical language used to define r\(_1\)-r\(_n\) in
(\ref{fsg}) thus constitutes a constraint definition language because it
explicitly defines the possible surface well-formedness constraints as
being those which forbid an ill-formed piece of a structure (a
substructure).

A FSC combines the OT representation of surface markedness (using *)
with the logical language for forbidding a substructure, like r\(_1\) in
(\ref{fsg}). A forbidden substructure grammar thus consists of the set
of surface markedness constraints that rule out ill-formed
substructures, i.e.~FSCs. FSCs serve as a type of phonotactic
restriction such that \enquote{well-formedness is based on contiguous
structures of a specific size} (Jardine, 2017, p. 3). One can use FSCs
as a definition of locality because they refer to elements within a
structure that are connected by either an ordering or association
relation. A phonological pattern is thus local if it can be described
with FSCs because it can be captured by referring to a subset of the
elements within structures and their connections. Jardine (2017) uses
FSCs to show that attested tone patterns are local in this way. This
qualifying paper will utilize FSCs over multi-tiered ARs to show that
vowel harmony patterns are local in the same way.

In addition, this qualifying paper will show that it is necessary to
restrict the expressivity of FSCs by excluding word boundaries from the
set of representations that can occur in multi-tiered ARs. Such a
restriction prevents FSCs over multi-tiered ARs from explicitly
restricting forbidden substructures to word edges, which distinguishes
the FSCs for attested vowel harmony patterns in Akan, Finnish, and
Turkish from the logically possible but unattested sour grapes pattern.

\subsection{Assimilation Mechanisms}\label{assimilation-mechanisms}

Vowel harmony has previously been analyzed as an assimilatory process
that results in multiple vowels being associated to the same feature on
the surface, but this qualifying paper analyzes only surface
representations of vowel harmony. In the vowel harmony literature, the
term \enquote{spreading} has generally referred to an assimilatory
process that transforms underlying ARs with underspecified vowels into
surface ARs in which at least some vowels are associated to a single
feature. On the surface, the result of a spreading process over ARs is a
structure in which a single feature is associated to multiple vowels on
the segmental tier, as in (\ref{akan.ex}) above. The surface result of
an agreement process over ARs is a structure in which two non-successive
vowel features on the same tier have identical binary values, as in the
simplified AR of a Finnish word in (\ref{agree}). In this paper, the
term \enquote{spreading} will refer to the resulting multiple
association of features rather than the process that derives such
structures. Similarly, \enquote{agreement} is used here to refer to
surface ARs with non-successive identical features, as in (\ref{agree}).
This qualifying paper will show that both spreading and agreement ARs
can represent vowel harmony patterns on the surface, and both kinds of
assimilation are captured by a theory of markedness based in FSCs.

\begin{exe}
\ex Agreement \label{agree} \\
  \begin{tikzpicture}[baseline=(current bounding box.north)]
  \matrix [matrix of nodes, row sep=2.5ex, column sep=2.25ex, nodes={text height=1em, text depth=0.5em}]
  {
(a)      & |(a)| +back &          & |(b)| -back &          & |(c)| +back & \\
|(c1)| r & |(1)| u     & |(c2)| v & |(2)| e     & |(c3)| t & |(3)| a     &  \\
  };
  \draw (a.south) -- (1.north);
  \draw (b.south) -- (2.north);
  \draw (c.south) -- (3.north);
  \draw[black,->] (a) -- (b);
  \draw[black,->] (b) -- (c);
  \draw[black,->] (c1) -- (1);
  \draw[black,->] (1) -- (c2);
  \draw[black,->] (c2) -- (2);
  \draw[black,->] (2) -- (c3);
  \draw[black,->] (c3) -- (3);
  \end{tikzpicture}
\end{exe}

It will be shown that vowel feature assimilation patterns that result
from both spreading and agreement are local because they are captured by
FSCs. FSCs are markedness constraints that represent the phonotactic
restrictions of a language and can further demonstrate the expressive
power of a particular representation. This paper will use FSCs to
capture both spreading and agreement surface patterns. Thus vowel
harmony can be considered a single set of patterns despite being derived
by different assimilatory processes because all surface vowel harmony
patterns are generated by a single theory of markedness.

\section{Neutral vowels}\label{neutral-vowels}

In languages that exhibit vowel harmony patterns, vowels are described
as either undergoing harmony or remaining neutral. Traditional accounts
of vowel harmony have identified two categories of neutral vowels:
blocking and transparent vowels (van der Hulst \& Smith, 1986). A vowel
is said to block harmony when the vowels on either side do not have to
share the same feature. A vowel is said to be transparent when the
vowels around it have the same feature, but the transparent vowel does
not share that feature. In other words, harmony appears to skip over
transparent vowels.

\subsection{Blocking vowels}\label{blocking-vowels}

An example of vowels that block harmony is found with ATR harmony in
Akan (Clements, 1976). The Akan vowel inventory, in Table
\ref{akan_vowels}, consists of ten vowels with two main featural
distinctions: \(\pm\) ATR and \(\pm\) low. There are two +low vowels,
{[}\textipa{3}{]} and {[}a{]}, +ATR and -ATR, respectively. All other
vowels are considered -low and distinguished by ATR such that the +ATR
vowels are {[}i, e, u, o{]} and the -ATR vowels are
{[}\textipa{I, E, U, O}{]}.

\begin{table}[h]
  \caption{Akan Vowels}
  \begin{tabular}{c|c|c}
       & +ATR        & -ATR       \\\hline\hline
  -low & i           & \textipa{I}\\ \cline{2-3}
       & u           & \textipa{U}\\ \cline{2-3}
       & e           & \textipa{E}\\ \cline{2-3}
       & o           & \textipa{O}\\\hline
  +low & \textipa{3} & a\\\hline
  \end{tabular}
  \label{akan_vowels}
\end{table}

The harmony generalization is that if a word contains a sequence of -low
vowels, then those vowels will also share the same ATR feature
(Clements, 1976). For example, the words in (\ref{akan_-low}) contain
only -low vowels, which are also all either +ATR or -ATR.

\begin{exe}
\ex{-low vowels share an ATR feature value}\label{akan_-low}
\begin{xlist}
  \ex tie `listen'
  \ex obejii `he came and removed it'
  \ex \textipa{O}b\textipa{E}j\textipa{E}\textipa{I}  `he came and did it'
  \ex wubenum\textraiseglotstop `you will suck it'
  \ex w\textipa{U}b\textipa{E}n\textipa{U}m\textraiseglotstop `you will drink it'
  \end{xlist}
\end{exe}

The surface requirement that -low vowels share the same ATR feature can
also be written as a FSC, which forbids two vowels associated to the
same -low feature from being associated to different ATR features, as in
(\ref{fsc.akana}). The ordering relation on the ATR tier in
(\ref{fsc.akana}) is omitted because the + or - values of the two ATR
features are irrelevant for this constraint, as long as they differ. The
ordering relation on the segmental tier of this FSC is also omitted and
the reason will be made clear by the example in (\ref{akan.obeijii}).

\begin{exe}
\ex \label{fsc.akana}
  \begin{tikzpicture}[baseline=(current bounding box.north)]
  \matrix [matrix of nodes, row sep=2.5ex, column sep=2.25ex, nodes={text height=1em, text depth=0.5em}] 
  {
$\ast$ & |(a)| +ATR & |(f)| -ATR \\
       & |(c)| V    & |(h)| V \\
       & |(e)| -low &  \\
  };
  \draw (a.south) -- (c.north);
  \draw (f.south) -- (h.north);
  \draw foreach \x in {c, h} {(\x.south) -- (e.north)};
  \end{tikzpicture}
\end{exe}

\begin{exe}
\ex{[obejii] `he came and removed it'}\label{akan.obeijii} \\
  \begin{tikzpicture}[baseline=(current bounding box.north)]
  \matrix [matrix of nodes, row sep=4ex, column sep=2.25ex, nodes={text height=1em, text depth=0.5em}] 
  {
(a) & |(a)| +ATR  &         &         &         &         &         & (b) * & |[red](i)|  \textbf{+ATR}  &         &         &         & |[red](j)| \textbf{-ATR}    & \\
    & |(b)| o     & |(c)| b & |(d)| e & |(e)| j & |(f)| i & |(g)| i &       & |(k)| o     & |(l)| b & |[red](m)| \textbf{e} & |(n)| j & |[red](o)| \textbf{\textipa{I}} & |(p)| \textipa{I} \\
    & |(h)| -low  &         &         &         &         &         &       & |[red](q)| \textbf{-low} & \\
  };
  \draw foreach \x in {b, d, f, g} {(a.south) -- (\x.north)};
  \draw foreach \x in {b, d, f, g} {(\x.south) -- (h.north)};
  \draw (i.south) -- (k.north);
  \draw[red,thick] (i.south) -- (m.north);
  \draw[red,thick] (j.south) -- (o.north);
  \draw (j.south) -- (p.north);
  \draw foreach \x in {k, p} {(\x.south) -- (q.north)};
  \draw[red,thick] foreach \x in {m, o} {(\x.south) -- (q.north)};
  \draw[black,->] (b) -- (c);
  \draw[black,->] (c) -- (d);
  \draw[black,->] (d) -- (e);
  \draw[black,->] (e) -- (f);
  \draw[black,->] (f) -- (g);
  \draw[black,->] (i) -- (j);
  \draw[black,->] (k) -- (l);
  \draw[black,->] (l) -- (m);
  \draw[black,->] (m) -- (n);
  \draw[black,->] (n) -- (o);
  \draw[black,->] (o) -- (p);
  \end{tikzpicture}
\end{exe}

The AR for the grammatical Akan word {[}obeijii{]} \enquote{he came and
removed it} is shown in (\ref{akan.obeijii}a). Here a single +ATR and a
single -low feature are each associated to each vowel within the word,
demonstrating full ATR and low harmony. On the other hand, the
hypothetical Akan word, {[}obej\textipa{II}{]}, represented in
(\ref{akan.obeijii}b) is ungrammatical because it demonstrates full -low
harmony, but does not demonstrate full ATR harmony; so, the AR in
(\ref{akan.obeijii}b) contains the forbidden structure of
(\ref{fsc.akana}), shown in bold and red.

However, in traditional vowel harmony terms the presence of a +low vowel
blocks the rightward spread of ATR, some examples are shown in
(\ref{akan_+low}). Translating this to the static surface
representations assumed here, two -low vowels must be associated to the
same ATR feature, but if a +low vowel intervenes they can be associated
to different ATR features. The representation of (\ref{akan_+low}a)
exemplifies this pattern and is shown in (\ref{ex.akan+low}).

\begin{exe}
  \ex{Vowels on either side of +low can have different ATR features}\label{akan_+low}
  \begin{xlist}
    \ex p\textipa{I}r\textipa{3}ko  `pig'
    \ex obisa\textipa{I} `he asked'
    \ex m\textipa{I}k\textipa{O}k\textipa{3}ri  `I go and weight it'
    \ex okog\textsuperscript{w}ar\textipa{I}\textraiseglotstop `he goes and washes'
  \end{xlist}
\end{exe}

\begin{exe}
\ex{[p\textipa{I}r\textipa{3}ko] `pig'}\label{ex.akan+low} \\
\begin{tikzpicture}[baseline=(current bounding box.north)]
  \matrix [matrix of nodes, 
          row sep=2.5ex, column sep=2.25ex,
          nodes={text height=1em, text depth=0.5em}] 
  {
        & |(a)| -ATR        &         &                   &         & |(b)| +ATR\\
|(c)| p & |(d)| \textipa{I} & |(e)| r & |(f)| \textipa{3} & |(g)| k & |(h)| o\\
        & |(i)| -low        &         & |(j)| +low        &         & |(k)| -low\\
  };
  \draw foreach \x in {d, f} {(a.south) -- (\x.north)};
  \draw (b.south) -- (h.north);
  \draw (d.south) -- (i.north);
  \draw (f.south) -- (j.north);
  \draw (h.south) -- (k.north);
  \draw[black,->] (a) -- (b);
  \draw[black,->] (c) -- (d);
  \draw[black,->] (d) -- (e);
  \draw[black,->] (e) -- (f);
  \draw[black,->] (f) -- (g);
  \draw[black,->] (g) -- (h);
  \draw[black,->] (i) -- (j);
  \draw[black,->] (j) -- (k);
  \end{tikzpicture}
\end{exe}

\noindent Crucially, the AR in (\ref{ex.akan+low}) does not contain the
FSC from (\ref{fsc.akana}). While the AR for {[}p\textipa{Ir3}ko{]}
\enquote{pig} does contain two vowels associated to a -low feature and
two different ATR features, they are each separately concatenated to the
intervening {[}\textipa{3}{]} vowel, which is associated to a +low
feature. So, the surrounding vowels are associated to two separate -low
features on the surface and thus the AR satisfies FS and the NCC.
Because the forbidden structure is not present {[}p\textipa{Ir3}ko{]}
\enquote{pig} is grammatical.

In summary, the vowel harmony pattern with blocking vowels in Akan can
be captured using the FSC in (\ref{fsc.akana}), which does not refer to
the successor relation on any tier. Akan vowel harmony could thus be
considered local because the FSC that captures the pattern need only
refer to the associations between vowels and features. The next section
outlines a vowel harmony pattern with transparent vowels.

\subsubsection{Surface spreading is
local}\label{surface-spreading-is-local}

Some previous analyses of vowel harmony assume that all harmony patterns
result from a single assimilation process: feature spreading. Feature
spreading is generally considered to be a transformation from an
underlying representation in which a single feature is associated to a
single vowel into a surface representation with multiple vowels
associated to the same feature, as in (\ref{spread}). In other words,
feature spreading maps an underspecified underlying AR onto a fully
specified surface AR with multiple association.

\begin{exe}
\ex Surface spreading \label{spread} \\
\begin{tikzpicture}[baseline=(current bounding box.north)]
  \matrix [matrix of nodes, 
          row sep=2.5ex, column sep=2.25ex,
          nodes={text height=1em, text depth=0.5em}] 
  {
        & |(1)| -ATR &         & &           & &         & |(3)| -ATR & \\
|(a)| t & |(b)| i    & |(c)| e & & $\mapsto$ & & |(d)| t & |(e)| i    & |(f)| e \\
        & |(2)| -low &         & &           & &         & |(4)| -low & \\
  };
  \draw (1.south) -- (b.north);
  \draw (b.south) -- (2.north);
  \draw foreach \x in {e, f} {(3.south) -- (\x.north)};
  \draw foreach \x in {e, f} {(\x.south) -- (4.north)};
  \draw[black,->] (a) -- (b);
  \draw[black,->] (b) -- (c);
  \draw[black,->] (d) -- (e);
  \draw[black,->] (e) -- (f);
  \end{tikzpicture}
\end{exe}

This paper focuses only on surface representations and Akan provides an
example of a pattern in which vowel harmony assimilation is represented
by spreading ARs. The surface spreading ARs used throughout this paper
consist of a single feature that is associated to multiple vowels. Akan
provides an example of a classic spreading pattern, in which an initial
vowel feature (ATR) is associated to all the vowels in a word to the
left of a +low blocking vowel, as shown in (\ref{akan.obeijii}a).

The analysis of Akan provided here demonstrates that spreading ARs are
local on the surface. Here locality means that spreading ARs consist of
a domain defined by a single ATR feature node, they must include a
contiguous span of vowels, but they are not bounded in length, as in
(\ref{akan.obeijii}a); or when two different ATR features are present,
one succeeds the other regardless of how many vowels are associated to
each. In addition, the FSC posited for Akan is able to capture the Akan
ATR harmony pattern for words with and without blocking vowels.

\subsection{Transparent vowels}\label{transparent-vowels}

Finnish provides an example of backness harmony with four transparent
vowels. The Finnish vowel inventory in Table \ref{finnish_vowels}
consists of 16 vowels with contrastive length and three main featural
distinctions: \(\pm\) back, \(\pm\) low, and \(\pm\) round (Ringen \&
Heinamaki, 1999; Välimaa-Blum, 1986). The four vowels transparent to
backness harmony, \textipa{[i, i:, e, e:]}, are all {[}-back, -round,
-low{]}. Of the harmonizing vowels
\textipa{[y, y:, u, u:, \o, \o:, o, o:]} are all +round and -low while
\textipa{[\ae, \ae:, A, A:]} are all +low and -round. The +back vowels
are \textipa{[u, u:, o, o:, A, A:]} and the -back vowels are
\textipa{[i, i:, e, e:, y, y:, \o, \o:, \ae, \ae:]}. The difference
between harmonizing and transparent Finnish vowels is characterized by
low and round feature values. Transparent vowels are all {[}-low,
-round{]} and thus harmonizing vowels have a positive value for the low
and/or round feature.

\begin{table}[h]
  \caption{Finnish Vowels}
  \begin{tabular}{c|c|c|c|c}
       & -round          & \multicolumn{2}{l|}{+round} &  \\\hline\hline
  -low & \textipa{i, i:} & \textipa{y, y:}             & \textipa{u, u:} \\\cline{2-4}
       & \textipa{e, e:} & \textipa{\o, \o:}           & \textipa{o, o:} \\\hline\cline{3-5}
  +low &                 & \textipa{\ae, \ae:}         & \textipa{A, A:} & -round\\\hline\hline
                         & \multicolumn{2}{l|}{-back}  & +back \\\hline
  \end{tabular}
  \label{finnish_vowels}
\end{table}

The Finnish harmony generalization is that all of the harmonizing vowels
in a root will share the same back feature with each other and
harmonizing suffix vowels will share the same back feature with the
harmonizing root-final vowel (Nevins, 2010; Ringen \& Heinamaki, 1999;
van der Hulst, 2017; Välimaa-Blum, 1986). Since the same harmony
generalization holds for both root and suffix vowels the Finnish
generalization can also be stated as two harmonizing vowels must share
the same back feature. For example, the words in
(\ref{finnish_-neutral}) contain only +round or +low vowels, which are
also either all +back or all -back.

\begin{exe}
\ex{harmonizing vowels share a back feature value}\label{finnish_-neutral}
\begin{xlist}
  \ex \textipa{p\o yt\ae} `table'
  \ex \textipa{k\ae nt\ae:} `turn'
  \ex \textipa{tyk\ae t\ae} `like'
  \ex \textipa{poutA} `fine weather'
  \ex \textipa{murtA:} `break'
  \ex \textipa{kokAtA} `cook'
  \end{xlist}
\end{exe}

Transparent vowels, however, do not block or undergo harmony so in the
Finnish words in (\ref{finnish_trans}) +back harmony appears to skip
over the {[}-back, -round, -low{]} vowels \textipa{[i, i:, e, e:]}. The
novel contribution of the current analysis is to treat transparent
vowels in the same way as harmonizing vowels; the FSCs posited in this
section are able to generate the Finnish pattern without
underspecification of back features.

\begin{exe}
\ex{back harmony skips over transparent vowels}\label{finnish_trans}
\begin{xlist}
  \ex \textipa{ruvetA} `start'
  \ex \textipa{tuoliA} `chair'
  \ex \textipa{lukeA} `read (inf.)'
  \ex \textipa{kAuneus} `beauty'
  \ex \textipa{nAivius} `naiveness'
  \ex \textipa{kotikAs} `cozy'
  \end{xlist}
\end{exe}

The surface requirement that +round and +low vowels share the same back
feature can also be stated negatively as a constraint that forbids
either a +round or a +low vowel from being associated to a different
preceeding back feature. Together, the four FSCs in (\ref{fsc.finnish})
generate this negative constraint and the Finnish vowel harmony pattern.
The ordering relation on the segmental tier of the FSCs is omitted
because the vowels can have consonants between them, as in
(\ref{finnish.pouta}). The ordering relation on the back tier, however,
is crucial in order to allow transparency of certain -back vowels.

\begin{exe}
\ex \label{fsc.finnish}
  \begin{tikzpicture}[baseline=(current bounding box.north)]
  \matrix [matrix of nodes, row sep={3.45em,between origins}, column sep={2.25em,between origins}, nodes={text height=1em, text depth=0.5em}] 
  {
(a)  & * & |(a)| +back &            & |(b)| -back & & (b) & * & |(1)| +back &  & |(2)| -back  \\
     & &               &            & |(d)| V     & &     &   &             &  & |(6)| V      \\
     & &               & |(e)| +low &             & &     &   &             &  & \\
     & &               &            &             & &     &   &             &  & |(9)| +round \\
(c)  & *          & |(f)| -back & & |(g)| +back & & (d)  & * & |(10)| -back  & & |(11)| +back \\
     &            & |(h)| V     & &             & &      &   & |(12)| V      & & \\
     & |(j)| +low &             & &             & &      &   &               & & \\
     &            &             & &             & &      &   & |(14)| +round & & \\
  };
  \draw (b.south) -- (d.north);
  \draw (d.south) -- (e.north);
  \draw[black,->] (a) -- (b);
  \draw (2.south) -- (6.north);
  \draw (6.south) -- (9.north);
  \draw[black,->] (a) -- (b);
  \draw[black,->] (1) -- (2);
  \draw (f.south) -- (h.north);
  \draw (h.south) -- (j.north);
  \draw[black,->] (f) -- (g);
  \draw (10.south) -- (12.north);
  \draw (12.south) -- (14.north);
  \draw[black,->] (10) -- (11);
  \end{tikzpicture}
\end{exe}

The ARs in (\ref{finnish.pouta}) illustrate how the Finnish FSCs rule
out ungrammatical disharmonic words. The AR for the grammatical Finnish
word \textipa{[poutA]} `fine weather', shown in (\ref{finnish.pouta}a),
contains both a +round and a +low non-initial vowel as well as a single
+back feature, which demonstrates full back harmony. The hypothetical
Finnish word, \textipa{[pout\ae]} in (\ref{finnish.pouta}b), however,
contains the forbidden structure of (\ref{fsc.finnish}a) in bold and
red. In (\ref{finnish.pouta}b) the final vowel does not harmonize with
the penultimate vowel because they are associated to different back
features. \newpage

\begin{exe}
\ex{\textipa{[poutA]} `fine weather'}\label{finnish.pouta} \\
  \begin{tikzpicture}[baseline=(current bounding box.north)]
  \matrix [matrix of nodes, row sep={3.45em,between origins}, column sep={2.25em,between origins}, nodes={text height=1em, text depth=0.5em}] 
  {
(a) &            & |(a)| +back &         &             &                   & & (b) * &               & |[red](n)| \textbf{+back} &                        &         &  |[red](o)| \textbf{-back} \\ 
    & |(b)| p    & |(c)| o     & |(d)| u & |(e)| t    & |(f)| \textipa{A} & &        & |(p)| p       & |(q)| o                   & |(r)| u  & |(s)| t                  & |[red](t)| \textbf{\ae}\\
    & |(l)| -low &             &         & |(m)| +low &                   & &        & |(z)| -low    &                           &                        & |[red](1)| \textbf{+low} & \\
    &              & |(j)| +round  &         &             & |(k)| -round & &       &               & |(x)| +round     &                        &          & |(y)| -round             & \\
  };
  \draw foreach \x in {c, d, f} {(a.south) -- (\x.north)};
  \draw foreach \x in {c, d} {(\x.south) -- (j.north)};
  \draw foreach \x in {c, d} {(\x.south) -- (l.north)};
  \draw (f.south) -- (k.north);
  \draw (f.south) -- (m.north);
  \draw foreach \x in {q, r} {(n.south) -- (\x.north)};
  \draw[thick,red] (o.south) -- (t.north);
  \draw foreach \x in {q, r} {(\x.south) -- (x.north)};
  \draw foreach \x in {q, r} {(\x.south) -- (z.north)};
  \draw (t.south) -- (y.north);
  \draw[thick,red] (t.south) -- (1.north);
  \draw[black,->] (b) -- (c);
  \draw[black,->] (c) -- (d);
  \draw[black,->] (d) -- (e);
  \draw[black,->] (e) -- (f);
  \draw[black,->] (j) -- (k);
  \draw[black,->] (l) -- (m);
  \draw[thick,red,->] (n) -- (o);
  \draw[black,->] (p) -- (q);
  \draw[black,->] (q) -- (r);
  \draw[black,->] (r) -- (s);
  \draw[black,->] (s) -- (t);
  \draw[black,->] (x) -- (y);
  \draw[black,->] (z) -- (1);
  \end{tikzpicture}
\end{exe}

Crucially, the behavior of the transparent vowels with respect to vowel
harmony in Finnish is captured by the four FSCs in (\ref{fsc.finnish})
without reference to underspecification of back features. For example,
the words in (\ref{finnish_trans}) all contain vowels with -back
features that follow +back vowels, but because the -back vowels are also
{[}-low, -round{]} the words are grammatical. The transparent vowels are
associated to all the same features as the harmonizing vowels and their
so-called transparency results from the interaction of the -back
features with -low and -round features, as shown in
(\ref{finnish.ruveta}). Because the Finnish FSCs only forbid
associations to certain back features when vowels are also either +low
or +round, the {[}-back, -low, -round{]} vowels are able to occur
anywhere within a word and do not affect the back feature values of
other vowels.

\begin{exe}
\ex{\textipa{\textipa{[ruvetA]}} `start'}\label{finnish.ruveta} \\
  \begin{tikzpicture}[baseline=(current bounding box.north)]
  \matrix [matrix of nodes, row sep={3.45em,between origins}, column sep={2.25em,between origins}, nodes={text height=1em, text depth=0.5em}] 
  {
(a) &            & |(a)| +back &         & |(b)| -back &            & |(c)| +back       & & (b) * &          & |[red](n)| \textbf{+back}           &          & |[red](o)| \textbf{-back}           &         & \\ 
    & |(d)| r    & |(e)| u     & |(f)| v & |(g)| e     & |(h)| t    & |(i)| \textipa{A} & &       & |(q)| r & |(r)| u & |(s)| v  & |(t)| e & |(u)| t & |[red](v)| \textbf{\textipa{\ae}}\\
    & |(j)| -low &             &         &             &            & |(k)| +low        & &       & |(w)| -low  &    &                       &          &                           & |[red](x)| \textbf{+low} & \\
    &            & |(l)| +round &         &            & |(m)| -round &       &           &       &          & |(y)| +round &           &          & |(z)| -round &   & \\
  };
  \draw (a.south) -- (e.north);
  \draw (b.south) -- (g.north);
  \draw (c.south) -- (i.north);
  \draw (n.south) -- (r.north);
  \draw (o.south) -- (t.north);
  \draw[thick,red] (o.south) -- (v.north);
  \draw foreach \x in {e, g} {(\x.south) -- (j.north)};
  \draw (e.south) -- (l.north);
  \draw (i.south) -- (k.north);
  \draw foreach \x in {g, i} {(\x.south) -- (m.north)};
  \draw foreach \x in {r, t} {(\x.south) -- (w.north)};
  \draw (r.south) -- (y.north);
  \draw (t.south) -- (z.north);
  \draw[thick,red] (v.south) -- (x.north);
  \draw (v.south) -- (z.north);
  \draw[black,->] (a) -- (b);
  \draw[black,->] (b) -- (c);
  \draw[thick,red,->] (n) -- (o);
  \draw[black,->] (d) --(e);
  \draw[black,->] (e) -- (f);
  \draw[black,->] (f) -- (g);
  \draw[black,->] (g) -- (h);
  \draw[black,->] (h) -- (i);
  \draw[black,->] (q) -- (r);
  \draw[black,->] (r) -- (s);
  \draw[black,->] (s) -- (t);
  \draw[black,->] (t) -- (u);
  \draw[black,->] (u) --(v);
  \draw[black,->] (j) -- (k);
  \draw[black,->] (w) -- (x);
  \draw[black,->] (l) -- (m);
  \draw[black,->] (y) -- (z);
  \end{tikzpicture}
\end{exe}

In (\ref{finnish.ruveta}a) the {[}u{]} and {[}\textipa{A}{]} vowels are
each associated to a +back feature, but not the same one. The {[}e{]}
vowel occurs between them and is associated to a -back feature. The two
+back features are also in a successor relation with the intervening
-back feature. The AR in (\ref{finnish.ruveta}a) is grammatical because
it satisfies FS, the NCC, and the OCP and the -back vowel is not
associated to a +low or a +round feature, so the AR does not violate any
of the FSCs in (\ref{fsc.finnish}). The AR in (\ref{finnish.ruveta}b),
on the other hand, contains a {[}-back, +low{]} vowel, and so
(\ref{fsc.finnish}a) is violated, as shown in bold and red. Despite
being separated by a transparent vowel, it is still necessary for the
suffix and root vowels to agree and the same FSCs that capture Finnish
back harmony in (\ref{finnish.ruveta}a) also enforce agreement across a
transparent vowel by marking words like (\ref{finnish.ruveta}b) as
ungrammatical.

In summary, the vowel harmony pattern with transparent vowels in Finnish
can be captured using the four FSCs in (\ref{fsc.finnish}). These FSCs
refer to the successor relation(s) on the back tier, which also
interacts with both the round and low feature tiers. Finnish vowel
harmony could thus be considered local because the FSCs that capture the
pattern refer to the associations between vowels and features and the
ordering between features.

\subsubsection{Surface agreement is
local}\label{surface-agreement-is-local}

Finnish exemplifies an assimilation mechanism that differs from
spreading. In Finnish there are grammatical words like
(\ref{finnish.ruveta}a), which contain two +back harmonizing vowels with
a transparent vowel between them. The NCC prevents Finnish from using
spreading ARs because a single +back feature cannot be associated to a
vowel across an intervening -back feature. Two +back features can occur
because they are not in a successor relation with each other, so ARs do
not violate the OCP. So, the assimilation between two vowels of a +back
feature in words like (\ref{finnish.ruveta}a) must be due to a mechanism
that differs from feature spreading because the two +back features are
not successive. In addition, the OCP allows multiple iterations of a
+back feature to occur as long as each is in a successor relation with
the intervening -back feature. In this paper, this other type of
assimilation is called \emph{agreement}. Agreement is represented on the
surface as an AR in which two non-successive features on a tier share a
value and the intervening feature on that tier has the opposite value,
as shown in (\ref{finnish.ruveta}) and (\ref{finnish.agree}).

\begin{exe}
\ex{\textipa{[mAisemiA]} `scenery.plural.partitive'}\label{finnish.agree} \\
\begin{tikzpicture}[baseline=(current bounding box.north)]
  \matrix [matrix of nodes, row sep={3.45em,between origins}, column sep={3.5em,between origins}, nodes={text height=1em, text depth=0.5em}] 
  {
           & |(a)| +back       &  |(b)| -back &            &              &         &            &  |(c)| +back \\ 
|(1)| m    & |(2)| \textipa{A} & |(3)| i      & |(4)| s    & |(5)| e      & |(6)| m & |(7)| i    & |(8)| \textipa{A} \\
|(d)| +low &                   & |(e)| -low   &            &              &          &           & |(f)| +low \\
           & |(g)| -round      &              &            &              &          &           &  \\
  };
  \draw (a.south) -- (2.north);
  \draw foreach \x in {3, 5, 7} {(b.south) -- (\x.north)};
  \draw (c.south) -- (8.north);
  \draw (2.south) -- (d.north);
  \draw foreach \x in {3, 5, 7} {(\x.south) -- (e.north)};
  \draw (8.south) -- (f.north);
  \draw foreach \x in {2, 3, 5, 7, 8} {(\x.south) -- (g.north)};
  \draw[black,->] (a) -- (b);
  \draw[black,->] (b) -- (c);
  \draw[black,->] (d) --(e);
  \draw[black,->] (e) -- (f);
  \draw[black,->] (1) -- (2);
  \draw[black,->] (2) -- (3);
  \draw[black,->] (3) -- (4);
  \draw[black,->] (4) -- (5);
  \draw[black,->] (5) -- (6);
  \draw[black,->] (6) -- (7);
  \draw[black,->] (7) -- (8);
\end{tikzpicture}

\end{exe}

The analysis of Finnish provided here demonstrates that agreement ARs
are local on the surface. Transparent vowels are associated to -back
features on the same tier as the back features to which all other vowels
associate, which eliminates the need for underspecification. Identical
+back features are connected via the successor relation to the -back
feature between them regardless of the number of vowels associated to
the intervening -back feature. In (\ref{finnish.agree}) more than one
transparent vowel is associated to a -back feature that intervenes
between two +back features. On the segmental tier it would appear that
two +back vowels, such as {[}\textipa{A}{]} and {[}o{]}, can be
separated by more than one -back vowel, but on the back feature tier the
+back and -back features are in a successor relation. The +back
agreement appears to skip over any number of transparent vowels because
they are all associated to a single -back feature, which is in a
successor relation with the agreeing +back features. The successor
relation on the back tier allows transparent vowels to be associated to
a feature on the same tier as harmonizing vowels, rather than being
underspecified. The grammatical AR in (\ref{finnish.agree}) also does
not violate the Finnish FSCs in (\ref{fsc.finnish}) and so the Finnish
FSCs are still able to capture the agreement pattern. Thus Finnish vowel
harmony demonstrates surface agreement and can be considered local.

\section{Morphologically-conditioned
harmony}\label{morphologically-conditioned-harmony}

\subsection{Turkish}\label{turkish}

Native Turkish words demonstrate two separate harmony patterns: back and
round harmony. In Turkish, a suffix vowel shares its back feature with
the root-final vowel, but it is debated whether or not Turkish also
utilizes back harmony within roots. In addition, a +high suffix vowel
shares its round feature with the root-final vowel. The vowel inventory
of Turkish in Table \ref{turkish_vowels} consists of eight vowels with
three main featural distinctions: \(\pm\) high, \(\pm\) back, \(\pm\)
round. In Turkish the +high -back vowels are {[}i, ü{]}, the +high +back
vowels are \textipa{[1, u]}, the -high -back vowels are {[}e, ö{]}, and
the -high +back vowels are {[}a, o{]}.

\begin{table}[h]
  \caption{Turkish Vowels}
  \begin{tabular}{cc|c|cc|c|c|cc}
        & & \multicolumn{2}{|l}{-back}  &        & \multicolumn{2}{|l}{+back} &             & \\\cline{1-7}\cline{1-7}
  +high & & i                           & ü      &                            & \textipa{1} & u \\\cline{1-7}
  -high & & e                           & ö      &                            & a           & o \\\cline{1-7}\cline{1-7}
        & & -round                      & +round &                            & -round      & +round \\
  \end{tabular}
  \label{turkish_vowels}
\end{table}

The Turkish back harmony generalization is that all suffix vowels share
the same back feature as the root-final vowel and the round harmony
generalization is that a high suffix vowel shares the same round feature
as the root-final vowel (Clements, 1976; Crothers \& Shibatani, 1980;
Nevins, 2010; Padgett, 2002; van der Hulst, 2017). For example, the
words in (\ref{turkish_back}) contain suffix vowels that have the same
back feature as the preceding root-final vowel. In addition, the high
suffix vowels in (\ref{turkish_back}b-e) have the same round feature as
the root-final vowel. Unlike in Finnish, Turkish non-final root vowels
and suffix vowels do not necessarily share the same features on the
surface, which makes it necessary to distinguish morphemes in ARs.
Throughout this section, root and suffix morphemes will be distinguished
by their position relative to a morpheme boundary, i.e.~roots are on the
left and suffixes on the right. In words with multiple suffixes, the
first or leftmost morpheme is the root and any morphemes to the right of
it are considered to be suffixes. In (\ref{turkish_back}), for example,
a morpheme boundary is represented by a large plus sign
`\textipa{\LARGE+}'.

\begin{exe}
\ex{Suffix vowels share a back feature with root-final vowels}\label{turkish_back} 
\begin{xlist}
  \ex \textipa{ip\LARGE+}ler `rope (Nom.pl)'
  \ex \textipa{köy\LARGE+}\textipa{ün} `village (Gen.sg)'
  \ex \textipa{el\LARGE+}i   `hand (Acc.sg)'
  \ex \textipa{k1z\LARGE+}\textipa{1n} `girl (Gen.sg)'
  \ex \textipa{son\LARGE+}u  `end (Acc.sg)'
  \ex \textipa{pul\LARGE+}lar `stamp (Nom.pl)'
\end{xlist}
\end{exe}

In addition, Turkish consists of grammatical words with disharmonic
roots, as in (\ref{turkish_disharmonic}). A lack of back harmony within
root words prevents root and suffix back harmony generalizations from
being collapsed into a single harmony pattern, as in Finnish. If back
harmony holds only between root-final and suffix vowels, but not within
roots, the back features associated to those vowels must also be
distinguished as either root or suffix features.

\begin{exe}
\ex{Turkish words with disharmonic roots}\label{turkish_disharmonic} 
\begin{xlist}
  \ex butik   `boutique'
  \ex bordür  `edge ornamentation'
  \ex kuvvet  `strength'
  \ex mezat   `auction'
  \ex tatil   `vacation'
\end{xlist}
\end{exe}

The Turkish back harmony pattern can thus be captured by an FSC that
forbids two successive back features on either side of a morpheme
boundary from having different values, as in (\ref{fsc.turkishback}). In
the FSC in (\ref{fsc.turkishback}a), the +back vowel and the -back vowel
must also be identifiable as the root and suffix vowels, respectively.
The successor ordering relation on the back tier ensures that the +back
vowel is to the left and the -back vowel is to the right. In addition,
the morpheme boundary must be represented on the back tier in order to
distinguish the root feature from the suffix feature. The same reasoning
holds for the FSC in (\ref{fsc.turkishback}b), but the root-final vowel
is associated to a -back and the suffix vowel is associated to a +back
feature.

\begin{exe}
\ex \label{fsc.turkishback}
  \begin{tikzpicture}[baseline=(current bounding box.north)]
  \matrix [matrix of nodes, row sep={3.45em,between origins}, column sep={3em,between origins}, nodes={text height=1em, text depth=0.5em}] 
  {
(a) * & |(a)| +back & |(b)| \LARGE+ & |(c)| -back & & (b) * & |(1)| -back & |(2)| \LARGE+ & |(3)| +back \\
  };
  \draw[black,->] (a) -- (b);
  \draw[black,->] (b) -- (c);
  \draw[black,->] (1) -- (2);
  \draw[black,->] (2) -- (3);
  \end{tikzpicture}
\end{exe}

\begin{exe}
\ex{[\textipa{ip\LARGE+}ler] `rope (Nom.pl)'}\label{turkish.ipler} \\
  \begin{tikzpicture}[baseline=(current bounding box.north)]
  \matrix [matrix of nodes, row sep={3.45em,between origins}, column sep={2.25em,between origins}, nodes={text height=1em, text depth=0.5em}] 
  {
(a) & |(a)| -back &              & |(3)| \LARGE+ &         & |(2)| -back &         & (b) * & |[red](h)| \textbf{-back} &         & |[red](1)| \textbf{\LARGE+} &         & |[red](i)| \textbf{+back} & \\ 
    & |(b)| i     & |(c)| p      & |(d)| \LARGE+ & |(e)| l & |(f)| e     & |(g)| r &       & |(j)| i     & |(k)| p & |(l)| \LARGE+               & |(m)| l & |(n)| a & |(o)| r \\
  };
  \draw (a.south) -- (b.north);
  \draw (2.south) -- (f.north);
  \draw (h.south) -- (j.north);
  \draw (i.south) -- (n.north);
  \draw (3.south) -- (d.north);
  \draw (1.south) -- (l.north);
  \draw[black,->] (a) -- (3);
  \draw[black,->] (3) -- (2);
  \draw[black,->] (b) -- (c);
  \draw[black,->] (c) -- (d);
  \draw[black,->] (d) -- (e);
  \draw[black,->] (e) -- (f);
  \draw[black,->] (f) -- (g);
  \draw[black,->] (j) -- (k);
  \draw[black,->] (k) -- (l);
  \draw[black,->] (l) -- (m);
  \draw[black,->] (m) -- (n);
  \draw[black,->] (n) -- (o);
  \draw[red,thick,->] (h) -- (1);
  \draw[red,thick,->] (1) -- (i);
  \end{tikzpicture}
\end{exe}

\noindent The Turkish word {[}\textipa{ip\LARGE+}ler{]} `rope (Nom.pl)'
illustrates full back harmony, as shown in (\ref{turkish.ipler}a): both
the root and suffix features are -back. The two vowels in
(\ref{turkish.ipler}a) are not separated by any other vowels, but are
associated to different -back features. Including the morpheme boundary
on the feature tier prevents the two -back features from being in a
successor relation with each other; and so both -back features are
represented in order to satisfy the NCC. The hypothetical word
{[}\textipa{ip\LARGE+}lar{]}, however, is ungrammatical because the root
and suffix vowels are associated to different back features and so the
AR in (\ref{turkish.ipler}b) contains the forbidden substructure of
(\ref{fsc.turkishback}b) in bold and red.

Turkish round harmony can also be written as an FSC, which forbids a
suffix vowel that is associated to a +high feature from also being
associated to a different round feature from the root-final vowel, as in
(\ref{fsc.turkishround}).

\begin{exe}
\ex \label{fsc.turkishround}
  \begin{tikzpicture}[baseline=(current bounding box.north)]
  \matrix [matrix of nodes, row sep={3.45em,between origins}, column sep={3.10em,between origins}, nodes={text height=1em, text depth=0.5em}] 
  {
(a) *        &               & |(a)| +high  & & & (b) *        &               & |(1)| +high \\
             &               & |(c)| V      & & &              &               & |(3)| V \\
|(d)| +round & |(f)| \LARGE+ & |(e)| -round & & & |(4)| -round & |(5)| \LARGE+ & |(6)| +round \\
  };
  \draw (a.south) -- (c.north);
  \draw (c.south) -- (e.north);
  \draw (1.south) -- (3.north);
  \draw (3.south) -- (6.north);
  \draw[black,->] (d) -- (f);
  \draw[black,->] (f) -- (e);
  \draw[black,->] (4) -- (5);
  \draw[black,->] (5) -- (6);
  \end{tikzpicture}
\end{exe}

\begin{exe}
\ex{[\textipa{köy\LARGE+}ün] `village (Gen.sg)'}\label{turkish.koyun} \\
  \begin{tikzpicture}[baseline=(current bounding box.north)]
  \matrix [matrix of nodes, row sep={3.45em,between origins}, column sep={3.25em,between origins}, nodes={text height=1em, text depth=0.5em}] 
  {
(a)     & |(a)| -high  &         & |(1)| \LARGE+ & |(g)| +high &          & (b) * & |(h)| -high                &         & |(3)| \LARGE+ & |[red](i)| \textbf{+high} & \\
|(b)| k & |(c)| ö      & |(d)| y & |(s)| \LARGE+ & |(e)| ü      & |(f)| n & |(j)| k & |(k)| ö      & |(l)| y & |(m)| \LARGE+              & |[red](n)| \textbf{i} & |(o)| n \\
        & |(p)| +round &         & |(4)| \LARGE+ & |(5)| +round &         &        & |[red](q)| \textbf{+round} &         & |[red](2)| \textbf{\LARGE+} & |[red](r)| \textbf{-round} \\         
  };
  \draw (a.south) -- (c.north);
  \draw (g.south) -- (e.north);
  \draw (c.south) -- (p.north);
  \draw (e.south) -- (5.north);
  \draw (h.south) -- (k.north);
  \draw[red,thick] (i.south) -- (n.north);
  \draw (k.south) -- (q.north);
  \draw[red,thick] (n.south) -- (r.north);
  \draw (1.south) -- (s.north);
  \draw (3.south) -- (m.north);
  \draw (s.south) -- (4.north);
  \draw (m.south) -- (2.north);
  \draw[black,->] (a) -- (1);
  \draw[black,->] (1) -- (g);
  \draw[black,->] (b) -- (c);
  \draw[black,->] (c) -- (d);
  \draw[black,->] (d) -- (s);
  \draw[black,->] (s) -- (e);
  \draw[black,->] (e) -- (f);
  \draw[black,->] (j) -- (k);
  \draw[black,->] (k) -- (l);
  \draw[black,->] (l) -- (m);
  \draw[black,->] (m) -- (n);
  \draw[black,->] (n) -- (o);
  \draw[red,thick,->] (q) -- (2);
  \draw[red,thick,->] (2) -- (r);
  \draw[black,->] (h) -- (3);
  \draw[black,->] (3) -- (i);
  \draw[black,->] (p) -- (4);
  \draw[black,->] (4) -- (5);
  \end{tikzpicture}
\end{exe}

\noindent The AR for the grammatical Turkish word
{[}\textipa{köy\LARGE+}ün{]} `village (Gen.sg)' shown in
(\ref{turkish.koyun}a) contains a -high root-final vowel and a +high
suffix vowel. Both vowels are associated to a +round feature, which
demonstrates full round harmony. The hypothetical Turkish word
{[}köy-in{]} in (\ref{turkish.koyun}b), on the other hand, contains the
forbidden structure of (\ref{fsc.turkishround}a) in bold and red because
it does not demonstrate full round harmony; the +high suffix vowel is
associated to a different round feature than the root-final vowel.

The analysis presented above captures both the Turkish back and round
harmony patterns with FSCs in which morpheme boundaries are represented
on all feature tiers. One critique of such an analysis could be that
morpheme boundaries can only be represented on the segmental tier and
not on feature tiers. Such an analysis would correctly rule out
disharmonic suffixes, and would also incorrectly rule out disharmonic
roots in Turkish. For example, if the morpheme boundary is removed from
the feature tiers, then the FSCs in (\ref{fsc.turkishback}) would look
like those in (\ref{fsc.turkish.nomorph}). The FSCs in
(\ref{fsc.turkish.nomorph}) forbid any two successive back features from
having different values without regards to morpheme boundaries.

\begin{exe}
\ex \label{fsc.turkish.nomorph}
  \begin{tikzpicture}[baseline=(current bounding box.north)]
  \matrix [matrix of nodes, row sep={3.45em,between origins}, column sep={3.5em,between origins}, nodes={text height=1em, text depth=0.5em}] 
    {
(a) * & |(a)| +back & |(c)| -back & & (b) * & |(1)| -back & |(3)| +back \\
  };
  \draw[black,->] (a) -- (c);
  \draw[black,->] (1) -- (3);
  \end{tikzpicture}
\end{exe}

On the surface, all Turkish suffix vowels are associated to the same
back feature as root-final vowels. While most Turkish suffixes are
monosyllabic, in a grammatical Turkish word with two suffixes the same
descriptive generalization holds. For example, in
{[}\textipa{k1z\LARGE+}\textipa{lar}\textipa{\LARGE+}\textipa{1n}{]}
`girls (gen.)' both suffix vowels are associated to the same back
feature on the surface, as shown in the AR (\ref{turkish.dprim}). The
word
{[}\textipa{k1z\LARGE+}\textipa{lar}\textipa{\LARGE+}\textipa{1n}{]}
contains the root {[}\textipa{k1z}{]} followed by two suffixes:
{[}lar{]} and {[}\textipa{1n}{]}. The grammatical AR in
(\ref{turkish.dprim}) thus demonstrates that all suffix vowels are
associated to the same back feature even when multiple suffixes are
present.

\begin{exe}
\ex \label{turkish.dprim} Turkish root-final and suffix vowels all associated to a single back feature \\
  \begin{tikzpicture}[baseline=(current bounding box.north)]
  \matrix [matrix of nodes, row sep={3.45em,between origins}, column sep={3.5em,between origins}, nodes={text height=1em, text depth=0.5em}]
  {
&         & |(a)| +back       &         &               &         &         &         &                        &                   & \\
& |(1)| k & |(2)| \textipa{1} & |(3)| z & |(4)| \LARGE+ & |(5)| l & |(6)| a & |(7)| r & |(pl)| \LARGE+ & |(8)| \textipa{1} & |(9)| n \\
  };
  \draw foreach \x in {2, 6, 8} {(a.south) -- (\x.north)};
  \draw[black,->] (1) -- (2);
  \draw[black,->] (2) -- (3);
  \draw[black,->] (3) -- (4);
  \draw[black,->] (4) -- (5);
  \draw[black,->] (5) -- (6);
  \draw[black,->] (6) -- (7);
  \draw[black,->] (7) -- (pl);
  \draw[black,->] (pl) -- (8);
  \draw[black,->] (8) -- (9);
  \end{tikzpicture}
\end{exe}

\begin{exe}
\ex \label{turkish.dishsuf}
  \begin{tikzpicture}[baseline=(current bounding box.north)]
  \matrix [matrix of nodes, row sep={3.45em,between origins}, column sep={3.5em,between origins}, nodes={text height=1em, text depth=0.5em}] 
    {

(a) & *        & |[red](d)| \textbf{+back}        &          &                &          &            &          &                          & |[red](g)| \textbf{-back} & \\
    & |(10)| k & |(11)| \textipa{1} & |(12)| z & |(13)| \LARGE+ & |(14)| l & |(15)| a                 & |(16)| r & |(pl)| \LARGE+           & |(17)| i                  & |(18)| n \\
(b) & *        & |(h)| +back        &          &                &          & |[red](j)| \textbf{-back} &          &                 & |[red](k)| \textbf{+back} & \\
    & |(19)| k & |(20)| \textipa{1} & |(21)| z & |(22)| \LARGE+ & |(23)| l & |(24)| e                              & |(25)| r & |(pl2)| \LARGE+ & |(26)| \textipa{1}        & |(27)| n \\
  };
  \draw foreach \x in {11, 15} {(d.south) -- (\x.north)};
  \draw (g.south) -- (17.north);
  \draw (h.south) -- (20.north);
  \draw (j.south) -- (24.north);
  \draw (k.south) -- (26.north);
  \draw[thick,red,->] (d) -- (g);
  \draw[black,->] (10) -- (11);
  \draw[black,->] (11) -- (12);
  \draw[black,->] (12) -- (13);
  \draw[black,->] (13) -- (14);
  \draw[black,->] (14) -- (15);
  \draw[black,->] (15) -- (16);
  \draw[black,->] (16) -- (pl);
  \draw[black,->] (pl) -- (17);
  \draw[black,->] (17) -- (18);
  \draw[black,->] (h) -- (j);
  \draw[thick,red,->] (j) -- (k);
  \draw[black,->] (19) -- (20);
  \draw[black,->] (20) -- (21);
  \draw[black,->] (21) -- (22);
  \draw[black,->] (22) -- (23);
  \draw[black,->] (23) -- (24);
  \draw[black,->] (24) -- (25);
  \draw[black,->] (25) -- (pl2);
  \draw[black,->] (pl2) -- (26);
  \draw[black,->] (26) -- (27);
  \end{tikzpicture}
\end{exe}

The ARs in (\ref{turkish.dishsuf}a) and (\ref{turkish.dishsuf}b), on the
other hand, violate the FSCs in (\ref{fsc.turkish.nomorph}a) and
(\ref{fsc.turkish.nomorph}b), respectively. In (\ref{turkish.dishsuf}a)
the first and second vowels are associated to a +back feature that
precedes a -back feature, shown in bold and red. Similarly, in
(\ref{turkish.dishsuf}b) the second vowel is associated to a -back vowel
that precedes a +back vowel, shown in bold and red. The second morpheme
is to the left of a boundary and the third morpheme is to the right of
the same boundary, but both are considered suffixes because they follow
an initial morpheme. Because the two suffixes do not share the same back
feature, (\ref{turkish.dishsuf}b) violates (\ref{fsc.turkish.nomorph}b)
despite both morphemes being suffixes; so back harmony holds between two
vowels in different suffixes as well as between a root-final and a
suffix vowel. Thus the FSCs in
(\ref{fsc.turkish.nomorph})\textemdash without morpheme boundaries on
feature tiers\textemdash do capture the suffix harmony pattern in
Turkish words with two suffixes.

However, Turkish also has grammatical words with disharmonic roots. The
FSCs in (\ref{fsc.turkish.nomorph}) do not discriminate between root and
suffix vowel features because there is no morpheme boundary on the
feature tier. A grammatical Turkish root like {[}tatil{]} `vacation'
would violate (\ref{fsc.turkish.nomorph}a) because the AR contains the
forbidden structure shown in bold and red in (\ref{turkish.tatil}).

\begin{exe}
\ex{[tatil] `vacation'} \label{turkish.tatil} \\
  \begin{tikzpicture}[baseline=(current bounding box.north)]
  \matrix [matrix of nodes, row sep={3.45em,between origins}, column sep={3.25em,between origins}, nodes={text height=1em, text depth=0.5em}] 
  {
        & |[red](1)| \textbf{+back} &         & |[red](2)| \textbf{-back} &         &               & \\ 
|(a)| t & |(b)| a                   & |(c)| t & |(d)| i                   & |(e)| l & \\
  };
  \draw (1.south) -- (b.north);
  \draw (2.south) -- (d.north);
  \draw[red,thick,->] (1) -- (2);
  \draw[black,->] (a) -- (b);
  \draw[black,->] (b) -- (c);
  \draw[black,->] (c) -- (d);
  \draw[black,->] (d) -- (e);
  \end{tikzpicture}
\end{exe}

\noindent While the AR in (\ref{turkish.tatil}) does violate the
hypothetical FSC in (\ref{fsc.turkish.nomorph}a), {[}tatil{]} is an
attested grammatical Turkish word. Because the FSC in
(\ref{fsc.turkish.nomorph}a) incorrectly marks an attested disharmonic
root as ungrammtical, (\ref{fsc.turkish.nomorph}) cannot be said to
capture the Turkish back harmony pattern. Alternatively, the FSC in
(\ref{fsc.turkishback}a) contains a morpheme boundary on the back
feature tier that intervenes between the two back features. Since a
disharmonic root like (\ref{turkish.tatil}) contains two different back
features in the same morpheme, it does not violate
(\ref{fsc.turkishback}a). For the same reason, the FSCs in
(\ref{fsc.turkishback}) predict that a disharmonic polysllabic suffix
would also be grammatical, but an initial search was unable to find any
such suffixes in Turkish. The FSCs in (\ref{fsc.turkishback}) must be
adopted to capture the Turkish back harmony pattern because they do not
mark attested disharmonic roots as ungrammatical. Adopting
(\ref{fsc.turkishback}) requires that morpheme boundaries are also
represented on feature tiers.

Adding a morpheme boundary to the feature tier allows the FSCs in
(\ref{fsc.turkishback}), repeated below in (\ref{turkish.final}), to
rule out words with a disharmonic suffix while still allowing words with
disharmonic roots. As shown below in
(\ref{turkish.final})-(\ref{kizlarin}), both {[}tatil{]} and
{[}\textipa{k1z\LARGE+}\textipa{lar}\textipa{\LARGE+}\textipa{1n}{]} are
captured by the same set of FSCs.

\begin{exe}
\ex Turkish FSCs \label{turkish.final}\\
  \begin{tikzpicture}[baseline=(current bounding box.north)]
  \matrix [matrix of nodes, row sep={3.45em,between origins}, column sep={3em,between origins}, nodes={text height=1em, text depth=0.5em}] 
  {
(a) $\ast$ & |(a)| +back & |(b)| \LARGE+ & |(c)| -back & & (b) $\ast$ & |(1)| -back & |(2)| \LARGE+ & |(3)| +back \\
  };
  \draw[black,->] (a) -- (b);
  \draw[black,->] (b) -- (c);
  \draw[black,->] (1) -- (2);
  \draw[black,->] (2) -- (3);
  \end{tikzpicture}
  
\ex {[tatil] `vacation'} \label{tatil} \\
  \begin{tikzpicture}[baseline=(current bounding box.north)]
  \matrix [matrix of nodes, row sep={3.45em,between origins}, column sep={3.25em,between origins}, nodes={text height=1em, text depth=0.5em}] 
  {
        & |(1)| +back &         & |(2)| -back &         &               & \\ 
|(a)| t & |(b)| a                   & |(c)| t & |(d)| i                   & |(e)| l & \\
  };
  \draw (1.south) -- (b.north);
  \draw (2.south) -- (d.north);
  \draw[black,->] (1) -- (2);
  \draw[black,->] (a) -- (b);
  \draw[black,->] (b) -- (c);
  \draw[black,->] (c) -- (d);
  \draw[black,->] (d) -- (e);
  \end{tikzpicture}

\ex {[\textipa{k1z\LARGE+}\textipa{lar}\textipa{\LARGE+}\textipa{1n}] `girls (gen.)'} \label{kizlarin} \\
  \begin{tikzpicture}[baseline=(current bounding box.north)]
  \matrix [matrix of nodes, row sep={3.45em,between origins}, column sep={3.5em,between origins}, nodes={text height=1em, text depth=0.5em}]
  {
        & |(a)| +back       &         & |(b)| \LARGE+ &         & |(c)| +back &         & |(d)| \LARGE+ & |(e)| +back       & \\
|(1)| k & |(2)| \textipa{1} & |(3)| z & |(4)| \LARGE+ & |(5)| l & |(6)| a     & |(7)| r & |(pl)| \LARGE+ & |(8)| \textipa{1} & |(9)| n \\
  };
  \draw (a.south) -- (2.north);
  \draw (c.south) -- (6.north);
  \draw (e.south) -- (8.north);
  \draw (b.south) -- (4.north);
  \draw (d.south) -- (pl.north);
  \draw[black,->] (a) -- (b);
  \draw[black,->] (b) -- (c);
  \draw[black,->] (c) -- (d);
  \draw[black,->] (d) -- (e);
  \draw[black,->] (1) -- (2);
  \draw[black,->] (2) -- (3);
  \draw[black,->] (3) -- (4);
  \draw[black,->] (4) -- (5);
  \draw[black,->] (5) -- (6);
  \draw[black,->] (6) -- (7);
  \draw[black,->] (7) -- (pl);
  \draw[black,->] (pl) -- (8);
  \draw[black,->] (8) -- (9);
  \end{tikzpicture}
\end{exe}

\noindent Again, the Turkish FSCs posited in this section are repeated
in (\ref{turkish.final}) above. Including morpheme boundaries on both
segmental and feature tiers allows these two FSCs to capture all of the
Turkish vowel harmony patterns discussed so far including words with
disharmonic roots, shown in (\ref{tatil}), and words with multiple
suffixes, shown in (\ref{kizlarin}).

In summary, Turkish demonstrates the necessity of adding a morpheme
boundary `\textipa{\LARGE+}' to the set of representations that can
occur on a feature tier. Because all root and suffix vowels do not
necessarily have to share the same features on the surface it is
necessary to distinguish them from one another. Identifying whether a
vowel is part of a root or a suffix is accomplished by ordering the
vowels relative to a morpheme boundary. However, since vowels are also
ordered with respect to consonants, it is necessary to include an
ordering relation on feature tiers so that the FSCs are connected. The
morpheme boundaries are projected onto each feature tier so that vowel
features are ordered relative to the boundary and can be distinguished
as belonging to a root or a suffix. In this way, the vowels are
indirectly ordered relative to each other via the ordering between their
features in addition to their association to those features. Turkish
vowel harmony can be considered local because both its back and round
harmony patterns are captured by the connected FSCs in
(\ref{fsc.turkishback}) and (\ref{fsc.turkishround}), respectively,
which refer to the associations between vowels and features as well as
the ordering relation between features and a morpheme boundary.

The present account of Turkish back harmony predicts the possibility of
disharmonic polysyllabic suffixes, but the initial survey did not reveal
any such suffixes in the language. A more in-depth review of attested
Turkish suffixes will be required to verify whether or not this
prediction is attested.

\subsubsection{Locality of Agreement}\label{locality-of-agreement}

Unlike Finnish, Turkish has both suffix vowel harmony and disharmonic
roots so the morphological domain of harmony must be restricted. The
analysis above adds a morpheme boundary to the set of representations
that can occur on both the feature and segmental tiers. Including a
primitive with morpheme boundaries on all tiers, as in (\ref{ipler}b),
increases the expressive power of ARs such that the possible domain of
feature spreading is restricted to a single morpheme by the successor
relation; however, feature assimilation between morphemes still occurs.

\begin{exe}
\ex{[\textipa{ip\LARGE+}ler] `rope (Nom.pl)'}\label{ipler} \\
  \begin{tikzpicture}[baseline=(current bounding box.north)]
  \matrix [matrix of nodes, row sep={3.45em,between origins}, column sep={2.25em,between origins}, nodes={text height=1em, text depth=0.5em}] 
  {
(a) & |(a)| -back &              & |(3)| \LARGE+ &         & |(2)| -back &  \\ 
    & |(b)| i     & |(c)| p      & |(d)| \LARGE+ & |(e)| l & |(f)| e     & |(g)| r & \\
(b) & |(4)| -back &                                     &         &                                        & |(5)| \LARGE+ &                                     &         &                                          & |(6)| -back &                                     & \\
    & |(h)| i     & \node {}; \draw(0, 0) circle (3pt); & |(p)| p & \node {}; \draw(0, 0) circle (3pt); & |(i)| \LARGE+ & \node {}; \draw(0, 0) circle (3pt); & |(j)| l & \node {}; \draw(0, 0) circle (3pt); & |(k)| e     & \node {}; \draw(0, 0) circle (3pt); & |(l)| r \\
  };
  \draw (a.south) -- (b.north);
  \draw (2.south) -- (f.north);
  \draw (3.south) -- (d.north);
  \draw[black,->] (a) -- (3);
  \draw[black,->] (3) -- (2);
  \draw[black,->] (b) -- (c);
  \draw[black,->] (c) -- (d);
  \draw[black,->] (d) -- (e);
  \draw[black,->] (e) -- (f);
  \draw[black,->] (f) -- (g);
  \draw (4.south) -- (h.north);
  \draw (5.south) -- (i.north);
  \draw (6.south) -- (k.north);
  \end{tikzpicture}
\end{exe}

\noindent The intervention of a morpheme boundary between identical
features in (\ref{turkish.ipler}a), repeated here in (\ref{ipler}a), is
derived via concatenation so that the two iterations of the -back
feature are not merged and remain on the surface. The assimilation of
-back must thus be due to agreement rather than spreading because the
surface AR in (\ref{ipler}a) contains two identical non-successive -back
features. The NCC is satisfied in (\ref{ipler}) because agreement allows
two iterations of the same feature on either side of the morpheme
boundary. The OCP is also satisfied because succesive elements on the
feature tier are distinct.

So far, vowel harmony has been shown to result from two different local
assimilation mechanisms. Spreading in Akan is local because it
associates multiple vocalic elements to a single feature--- thus
connecting distant vowels ---and utilizes the successor relation on
feature tiers. Agreement in both Finnish and Turkish also demonstrates
the local nature of vowel harmony assimilation. Computation over
agreement ARs is local because assimilation occurs over a finite
distance--- constrained by the successor relation between elements on
feature tiers.

While the concatenation of primitives is a universal process for
deriving the surface ARs used in this paper, each language determines
which primitives it makes use of. For example, Finnish utilizes the same
harmony pattern in roots and suffixes so a single set of FSCs can
capture the harmony pattern without referencing a morphological
boundary. Turkish, on the other hand, has disharmonic roots and suffixes
harmonize with the root-final vowel so Turkish FSCs must reference a
morphological boundary on feature tiers in order to distinguish suffix
features from root features. The difference between Finnish and Turkish
ARs with respect to the specification of morphological boundaries
results from the graph primitives that each language uses; Finnish does
not utilize morpheme boundary primitives, and Turkish utilizes
primitives with morpheme boundaries on both the segmental and feature
tiers. The next section shows how enriching the representation with
boundaries allows FSCs to also capture an unattested vowel harmony
pattern.

\section{Sour Grapes}\label{sour-grapes}

A phenomenon often discussed in autosegmental spreading literature is
the unattested, but logically possible pattern called sour grapes
(Lamont, 2018; McCarthy, 2011; Padgett, 1995). Sour grapes spreading is
described as a pattern in which a feature spreads throughout a word;
but, if the word contains a blocking segment no spreading occurs at all.
Sour grapes blockers could thus be considered to block spreading from
any distance. Lamont (2018) illustrates what a sour grapes pattern would
look like with nasal spreading, shown below in (\ref{nsg}).

\begin{exe}
\ex{Long distance blocking of local spreading, e.g. with nasal harmony (adapted from Lamont 2018)}\label{nsg}
\begin{xlist}
  \ex /wawa/ $\mapsto$ [wawa]
  \ex /mawa/ $\mapsto$ \textipa{[m\~a\~w\~a]}
  \ex /mawasa/ $\mapsto$ [mawasa]
\end{xlist}
\end{exe}

\noindent In (\ref{nsg}b) nasality spreads from an {[}m{]} onto each
segment to the right of it. In (\ref{nsg}c) an {[}s{]} is introduced,
which prevents nasality from spreading at all. Lamont (2018) further
shows that the nasal sour grapes pattern over two-tiered ARs must be
generated by a grammar that is more expressive than
ASL\textsuperscript{g\textsubscript{T}}. The nasal sour grapes pattern
does not meet the requirements for any of the subregular classes of
grammars that ASL\textsuperscript{g\textsubscript{T}} cuts across and so
Lamont (2018) posits that sour grapes must be generated by a more
expressive Regular grammar.

Following Lamont (2018), this section will evaluate whether or not FSCs
can be used to capture an unattested sour grapes pattern over
multi-tiered ARs of vowel harmony. It will be shown that a sour grapes
vowel harmony pattern can be described by FSCs over multi-tiered ARs
whether word boundaries are included in the set of representations
allowed on either the segmental or feature tiers.

\subsection{Sour grapes in vowel
harmony}\label{sour-grapes-in-vowel-harmony}

A parallel to the nasal sour grapes pattern can be drawn using the Akan
vowel harmony pattern discussed in section 2.1. In traditional
descriptions of Akan, the association of an ATR feature is said to
spread from an initial -low vowel onto all -low vowels to its right; for
example, the word \emph{\textipa{obisaI}} `he asked', shown in
(\ref{sgak}a) and (\ref{akan.sg}a), is grammatical in Akan. However,
+low vowels block Akan vowel harmony to their left; so an Akan-like sour
grapes word would have a +low vowel and the +ATR feature would not
spread to any vowel on the left of that +low vowel, as in (\ref{sgak}b)
and (\ref{akan.sg}b). Following (\ref{sgak}), the surface ARs in
(\ref{akan.sg}) show the difference between the full spreading harmony
in a word of Akan and the so-called long-distance blocking effect in a
related hypothetical word of the sour grapes pattern.

\begin{exe}
\ex{ATR harmony}\label{sgak} 
\begin{xlist}
  \ex Akan: /\textipa{obIsaI}/ $\mapsto$ [\textipa{obisaI}] `he asked'
  \ex sour grapes: /\textipa{obIsaI}/ $\mapsto$ [\textipa{obIsaI}]
\end{xlist}
\end{exe}

\begin{exe}
\ex{ATR harmony in Akan vs sour grapes}\label{akan.sg} \\
  \begin{tikzpicture}[baseline=(current bounding box.north)]
  \matrix [matrix of nodes, row sep={3.75em,between origins}, column sep={2.95em,between origins}, nodes={text height=1em, text depth=0.5em}] 
  {
(a) Akan   &           &         &           &             &                   &     & (b) Sour Grapes          &            &                   &            &           & \\
|(a)| +ATR &           &         &           & |(Ab)| -ATR &                   &     & |(b)| +ATR                   &            & |(f)| -ATR        &           &            & \\
|(1)| o    & |(co1)| b & |(2)| i & |(co2)| s & |(3)| a     & |(4)| \textipa{I} &     & |(5)| o                      & |(co3)| b  & |(6)| \textipa{I} & |(co4)| s & |(7)| a    & |(8)| \textipa{I} \\
|(c)| -low &           &         &           & |(f2)| +low & |(g)| -low        &     & |(d)| -low                   &            &                    &           & |(h)| +low & |(e)| -low \\
  };
  \draw (a.south) -- (1.north);
  \draw foreach \x in {1, 2} {(a.south) -- (\x.north)};
  \draw foreach \x in {1, 2} {(\x.south) -- (c.north)};
  \draw (3.south) -- (f2.north);
  \draw (4.south) -- (g.north);
  \draw foreach \x in {3, 4} {(Ab.south) -- (\x.north)};
  \draw (b.south) -- (5.north);
  \draw foreach \x in {6, 7, 8} {(f.south) -- (\x.north)};
  \draw foreach \x in {5, 6} {(\x.south) -- (d.north)};
  \draw (7.south) -- (h.north);
  \draw (8.south) -- (e.north);
  \draw[black,->] (a) -- (Ab);
  \draw[black,->] (b) -- (f);
  \draw[black,->] (1) -- (co1);
  \draw[black,->] (co1) -- (2);
  \draw[black,->] (2) -- (co2);
  \draw[black,->] (co2) -- (3);
  \draw[black,->] (3) -- (4);
  \draw[black,->] (5) -- (co3);
  \draw[black,->] (co3) -- (6);
  \draw[black,->] (6) -- (co4);
  \draw[black,->] (co4) -- (7);
  \draw[black,->] (7) -- (8);
  \draw[black,->] (c) -- (f2);
  \draw[black,->] (f2) -- (g);
  \draw[black,->] (d) -- (h);
  \draw[black,->] (h) -- (e);
  \end{tikzpicture}
\end{exe}

\noindent In (\ref{akan.sg}a) the +ATR feature has spread from the
initial -low vowel onto the -low vowel to its right and the +low vowel
is associated to a different ATR feature. However, in (\ref{akan.sg}b)
the penultimate +low vowel prevents the initial +ATR feature from
spreading, so the second vowel is associated to a -ATR feature.

On the surface, a word in an Akan-like language with sour grapes
(L\textsubscript{SG}) can be distinguished from Akan
(L\textsubscript{A}) based on the grammaticality of certain ARs. Both
L\textsubscript{A} and L\textsubscript{SG} include grammatical ARs with
full -ATR and -low harmony, as in (\ref{sgexar}b). The difference
between L\textsubscript{A} and L\textsubscript{SG} is that
L\textsubscript{SG} allows words with -ATR agreement and a final +low
vowel, shown in (\ref{sgexar}a), but L\textsubscript{A} does not.
Neither L\textsubscript{A} nor L\textsubscript{SG} allow words with ATR
agreement and full -low harmony, as in (\ref{sgexar}c). While Akan only
includes surface ARs with spreading, L\textsubscript{SG} contains a much
larger repertoire of assimilation patterns utilizing both spreading and
agreement, much like vowel harmony in general. A grammar that generates
L\textsubscript{SG} would thus need to distinguish (\ref{sgexar}a-b)
from (\ref{sgexar}c). In (\ref{sgexar}), the superscript `n' represents
any possible number of vowels that can occur in a given position with
the same featural associations. Using V\textsuperscript{n} suggests that
an AR will be (un)grammatical regardless of the word's length as long as
the given substructure is present. \newpage

\begin{exe}
\ex{Sour Grapes}\label{sgexar} \\
  \begin{tikzpicture}[baseline=(current bounding box.north)]
  \matrix [matrix of nodes, row sep={3.45em,between origins}, column sep={3.5em,between origins}, nodes={text height=1em, text depth=0.5em}]
{ 
(a) & |(1)| -ATR & |(2)| +ATR    & |(3)| -ATR &                 & (b) & |(4)| -ATR &              &         & \\
    & |(a)| V    & |(b)| V$^n$   & |(c)| V    & $\in$ L$_S$$_G$ &     & |(f)| V    & |(g)| V$^n$  & |(h)| V & $\in$ L$_S$$_G$, L$_A$ \\
    & |(d)| -low &             & |(e)| +low &                   &     & |(i)| -low      &              &        & \\
};
  \draw (1.south) -- (a.north);
  \draw foreach \x in {a, b} {(\x.south) -- (d.north)};
  \draw (2.south) -- (b.north);
  \draw (3.south) -- (c.north);
  \draw (c.south) -- (e.north);
  \draw foreach \x in {f, g, h} {(4.south) -- (\x.north)};
  \draw foreach \x in {f, g, h} {(\x.south) -- (i.north)};
  \draw[black,->] (1) -- (2);
  \draw[black,->] (2) -- (3);
  \draw[black,->] (d) -- (e);
  \end{tikzpicture}
  
  \begin{tikzpicture}[baseline=(current bounding box.north)]
  \matrix [matrix of nodes, row sep={3.45em,between origins}, column sep={3.5em,between origins}, nodes={text height=1em, text depth=0.5em}]
{ 
(c) & |(5)| -ATR & |(6)| +ATR    & |(7)| -ATR & \\
    & |(j)| V    & |(k)| V$^n$   & |(l)| V    & $\notin$ L$_S$$_G$, L$_A$ \\
    & |(m)| -low \\
};
  \draw (5.south) -- (j.north);
  \draw (6.south) -- (k.north);
  \draw (7.south) -- (l.north);
  \draw foreach \x in {j, k, l} {(\x.south) -- (m.north)};
  \draw[black,->] (5) -- (6);
  \draw[black,->] (6) -- (7);
  \end{tikzpicture}
\end{exe}

\noindent In other words, the L\textsubscript{SG} vowel harmony pattern
allows ATR agreement only when a +low vowel is present; otherwise only
full ATR and -low spreading harmony are grammatical. In order to rule
out a possible AR like (\ref{sgexar}c), a FSC would have to forbid a
substructure with more than one ATR feature, but no +low feature to the
right of -low, as in (\ref{fsc.1low}). Restricting the features to the
right of -low requires that FSCs make reference to final word
boundaries, which will be represented using the \textipa{\LARGE$\#$}
symbol.

\subsubsection{Boundaries on feature
tiers}\label{boundaries-on-feature-tiers}

Section 4.1 demonstrated, for Turkish, that morpheme boundaries must be
represented on feature tiers, and this same requirement can be extended
to word boundaries in a sour grapes pattern. As mentioned above,
L\textsubscript{SG} allows surface spreading ARs with full ATR and low
harmony in addition to surface agreement ARs only when a final +low
vowel is present. In order to restrict the occurrence of ATR agreement,
the FSCs in (\ref{sg.fsc}) forbid a structure with two different
successive ATR features when the -low feature precedes a final word
boundary.

\begin{exe}
\ex{Sour Grapes FSCs}\label{sg.fsc} \\
  \begin{tikzpicture}[baseline=(current bounding box.north)]
  \matrix [matrix of nodes, row sep={3.45em,between origins}, column sep={3.5em,between origins}, nodes={text height=1em, text depth=0.5em}]
{
(a) $\ast$  & |(a)| -ATR & |(b)| +ATR       & & (b) $\ast$ & |(e)| +ATR & |(f)| -ATR \\
            & |(1)| V    &                  & &            & |(2)| V    & \\
            & |(c)| -low & |(d)| \LARGE$\#$ & &            & |(g)| -low & |(h)| \LARGE$\#$ \\
};
  \draw (a.south) -- (1.north);
  \draw (1.south) -- (c.north);
  \draw[black,->] (a) -- (b);
  \draw[black,->] (c) -- (d);
  \draw (e.south) -- (2.north);
  \draw (2.south) -- (g.north);
  \draw[black,->] (e) -- (f);
  \draw[black,->] (g) -- (h);
    \end{tikzpicture}

  \ex{Ungrammatical L$_{SG}$ AR} \label{sg.forbid} \\
    \begin{tikzpicture}[baseline=(current bounding box.north)]
  \matrix [matrix of nodes, row sep={3.45em,between origins}, column sep={3.5em,between origins}, nodes={text height=1em, text depth=0.5em}]
{
|(a)| \LARGE$\#$ & |[red](b)| \textbf{+ATR} &         & |[red](c)| \textbf{-ATR} &         &        &                   & |(d)| \LARGE$\#$\\
|(1)| \LARGE$\#$ & |[red](2)| \textbf{o}    & |(3)| b & |(4)| \textipa{I}        & |(5)| s & |(6)| \textipa{E} & |(7)| \textipa{I} & |(8)| \LARGE$\#$ \\
|(e)| \LARGE$\#$ & |[red](f)| \textbf{-low} &         &                          &         &        &                & |[red](g)| \textbf{\LARGE$\#$}\\
};
  \draw[thick,red] (b.south) -- (2.north);
  \draw[thick,red] (2.south) -- (f.north);
  \draw foreach \x in {4, 6, 7} {(c.south) -- (\x.north)};
  \draw foreach \x in {4, 6} {(\x.south) -- (f.north)};
  \draw[black,->] (a) -- (b);
  \draw[thick,red,->] (b) -- (c);
  \draw[black,->] (c) -- (d);
  \draw[black,->] (1) -- (2);
  \draw[black,->] (2) -- (3);
  \draw[black,->] (3) -- (4);
  \draw[black,->] (4) -- (5);
  \draw[black,->] (5) -- (6);
  \draw[black,->] (6) -- (7);
  \draw[black,->] (7) -- (8);
  \draw[black,->] (e) -- (f);
  \draw[thick,red,->] (f) -- (g);
  \end{tikzpicture}
\end{exe}

L\textsubscript{SG} can be captured by FSCs when word boundaries are
represented on feature tiers. The FSC in (\ref{sg.fsc}) is able to
capture the constraint against a word-final -low feature. The AR of the
hypothetical L\textsubscript{SG} word {[}\textipa{obIseI}{]} in
(\ref{sg.forbid}), for example, is marked ungrammatical because it
contains the forbidden structure of (\ref{sg.fsc}) with a -low feature
succeeded by a final word boundary in bold and red.

\subsubsection{Boundaries only on segmental
tier}\label{boundaries-only-on-segmental-tier}

In addition, the argument could be made that word boundaries are
represented only on the segmental tier. In that case, the number of
features on a tier can be calculated by making reference to the
succession of a word boundary relative to a vowel and the associations
of vowels to features. The FSCs in (\ref{fsc.1low}) forbid a strutcure
with different ATR features in which the -low feature is associated to
the word-final vowel regardless of the number of vowels in the word. As
in Akan, the ordering relation on the ATR tier is excluded because the
same constraint holds regardless of whether -ATR precedes or succeeds
+ATR. The ordering relation is omitted between vowels because
word-medial consonants would make vowels not necessarily in a successor
relation with each other. Excluding the successor relation between
vowels also makes it possible for any number of vowels to occur between
those specified in the FSC without changing the grammaticality of the
word, as illustrated in (\ref{sg.1low}). The ARs in (\ref{sg.bound}) and
(\ref{sg.1low}) illustrate the difference between a grammatical and an
ungrammatical L\textsubscript{SG} word, captured by the FSCs in
(\ref{fsc.1low}).

\begin{exe}
\ex{Sour grapes FSCs with boundaries on segmental tier}\label{fsc.1low} \\
  \begin{tikzpicture}[baseline=(current bounding box.north)]
  \matrix [matrix of nodes, row sep={3.45em,between origins}, column sep={3.5em,between origins}, nodes={text height=1em, text depth=0.5em}]
{
(a) $\ast$ & |(a)| +ATR & |(b)| -ATR &                  & (b) $\ast$ & |(d)| -ATR & |(e)| +ATR & \\
           & |(1)| V    & |(2)| V    & |(3)| \LARGE$\#$ &            & |(4)| V    & |(5)| V    & |(7)| \LARGE$\#$\\
           & |(c)| -low &            &                  &            & |(f)| -low &            &                & \\
};
  \draw (a.south) -- (1.north);
  \draw (b.south) -- (2.north);
  \draw foreach \x in {1, 2} {(\x.south) -- (c.north)};
  \draw (d.south) -- (4.north);
  \draw (e.south) -- (5.north);
  \draw foreach \x in {4, 5} {(\x.south) -- (f.north)};
  \draw[black,->] (2) -- (3);
  \draw[black,->] (5) -- (7);
  \end{tikzpicture}

\ex{Grammatical L$_{SG}$ AR}\label{sg.bound} \\
  \begin{tikzpicture}[baseline=(current bounding box.north)]
  \matrix [matrix of nodes, row sep={3.45em,between origins}, column sep={3.5em,between origins}, nodes={text height=1em, text depth=0.5em}]
{
                 & |(b)| +ATR &         & |(c)| -ATR &      &         &            &                            & \\
|(1)| \LARGE$\#$ & |(2)| o    & |(3)| b & |(4)| \textipa{I} & |(5)| s & |(6)| a    & |(7)| \textipa{I} & |(8)| \LARGE$\#$ \\
                 & |(d)| -low &         &                   &         & |(e)| +low & |(f)| -low & \\
};
  \draw (b.south) -- (2.north);
  \draw foreach \x in {4, 6, 7} {(c.south) -- (\x.north)};
  \draw foreach \x in {2, 4} {(\x.south) -- (d.north)};
  \draw (6.south) -- (e.north);
  \draw (7.south) -- (f.north);
  \draw[black,->] (b) -- (c);
  \draw[black,->] (1) -- (2);
  \draw[black,->] (2) -- (3);
  \draw[black,->] (3) -- (4);
  \draw[black,->] (4) -- (5);
  \draw[black,->] (5) -- (6);
  \draw[black,->] (6) -- (7);
  \draw[black,->] (7) -- (8);
  \draw[black,->] (d) -- (e);
  \draw[black,->] (e) -- (f);
  \end{tikzpicture}

\ex{Ungrammatical L$_{SG}$ AR} \label{sg.1low} \\
  \begin{tikzpicture}[baseline=(current bounding box.north)]
  \matrix [matrix of nodes, row sep={3.65em,between origins}, column sep={3.5em,between origins}, nodes={text height=1em, text depth=0.5em}]
{
                 & |[red](b)| \textbf{+ATR}  &         & |[red](c)| \textbf{-ATR}        &         &                       &                                 & \\
|(1)| \LARGE$\#$ & |[red](2)| \textbf{o}     & |(3)| b & |(4)| \textipa{I} & |(5)| s & |(6)| \textipa{E} & |[red](7)| \textbf{\textipa{I}} & |[red](8)| \textbf{\LARGE$\#$} \\
                     & |[red](f)| \textbf-low &        &                                 &         &                       &                                 & \\
};
  \draw[thick,red] (b.south) -- (2.north);
  \draw[thick,red] (c.south) -- (7.north);
  \draw foreach \x in {4, 6} {(c.south) -- (\x.north)};
  \draw[thick,red] foreach \x in {2, 7} {(\x.south) -- (f.north)};
  \draw foreach \x in {4, 6} {(\x.south) -- (f.north)};
  \draw[black,->] (b) -- (c);
  \draw[black,->] (1) -- (2);
  \draw[black,->] (2) -- (3);
  \draw[black,->] (3) -- (4);
  \draw[black,->] (4) -- (5);
  \draw[black,->] (5) -- (6);
  \draw[black,->] (6) -- (7);
  \draw[thick,red,->] (7) -- (8);
  \end{tikzpicture}
\end{exe}

\noindent The FSCs in (\ref{fsc.1low}) allow the grammatical AR in
(\ref{sg.bound}) and mark (\ref{sg.1low}) ungrammatical because it
contains the forbidden substructure of (\ref{fsc.1low}b) in bold and
red. The difference between these two ARs is that (\ref{sg.bound})
contains a +low feature, but (\ref{sg.1low}) contains only a single -low
feature associated to the final vowel.

The FSCs in (\ref{fsc.1low}) necessarily include the successor relation
between a word-final vowel and the final word boundary, but Akan--- on
which L\textsubscript{SG} is based ---also includes words with final
consonants. Because consonants are not assoicated to vowel features, the
present theory has thusfar ignored them as irrelevant to vowel harmony
except to make vowels on the segmental tier non-successive. However, the
FSCs in (\ref{fsc.1low}) make reference to the successor relation
between a vowel and a final word boundary. If a word contains a
consonant between a vowel and the final word boundary (\ref{fsc.1low})
would not mark that word as ungrammatical in L\textsubscript{SG}. A word
with one or more final consonants could still contain the substructure
of (\ref{sgexar}c), which has been argued to be ungrammatical in
L\textsubscript{SG}. In order to rule out such ARs with final
consonants, one could posit an additional series of FSCs in which the
possible word-final consonants are represented with `C' and enumerated
succeeded by the final word boundary, as in (\ref{sgcons}).

\begin{exe}
\ex{Sour grapes FSCs with consonants and boundaries on segmental tier}\label{sgcons} \\
  \begin{tikzpicture}[baseline=(current bounding box.north)]
  \matrix [matrix of nodes, row sep={3.45em,between origins}, column sep={3.5em,between origins}, nodes={text height=1em, text depth=0.5em}]
{
(a) $\ast$ & |(a)| +ATR & |(b)| -ATR &          &                  & (b) $\ast$ & |(d)| +ATR & |(e)| -ATR &  |(g)| +ATR &          & \\
           & |(1)| V    & |(2)| V    & |(c1)| C & |(3)| \LARGE$\#$ &            & |(4)| V    & |(5)| V    & |(6)| V     & |(c2)| C & |(7)| \LARGE$\#$ \\
           & |(c)| -low &            &          &                  &            & |(f)| -low &           &             &          & \\
};
  \draw (a.south) -- (1.north);
  \draw (b.south) -- (2.north);
  \draw foreach \x in {1, 2} {(\x.south) -- (c.north)};
  \draw (d.south) -- ( 4.north);
  \draw (e.south) -- (5.north);
  \draw (g.south) -- (6.north);
  \draw foreach \x in {4, 5, 6} {(\x.south) -- (f.north)};
  \draw[black,->] (2) -- (c1);
  \draw[black,->] (c1) -- (3);
  \draw[black,->] (6) -- (c2);
  \draw[black,->] (c2) -- (7);
  \end{tikzpicture}
  
  \begin{tikzpicture}[baseline=(current bounding box.north)]
  \matrix [matrix of nodes, row sep={3.45em,between origins}, column sep={3.5em,between origins}, nodes={text height=1em, text depth=0.5em}]
{
(c) $\ast$ & |(a)| +ATR & |(b)| -ATR &          &          & \\
           & |(1)| V    & |(2)| V    & |(c1)| C & |(c2)| C & |(3)| \LARGE$\#$ & \\
           & |(c)| -low &            &          &          &                  & \\
};
  \draw (a.south) -- (1.north);
  \draw (b.south) -- (2.north);
  \draw foreach \x in {1, 2} {(\x.south) -- (c.north)};
  \draw[black,->] (2) -- (c1);
  \draw[black,->] (c1) -- (c2);
  \draw[black,->] (c2) -- (3);
  \end{tikzpicture}
  
  \begin{tikzpicture}[baseline=(current bounding box.north)]
  \matrix [matrix of nodes, row sep={3.45em,between origins}, column sep={3.5em,between origins}, nodes={text height=1em, text depth=0.5em}]
{
 (d) $\ast$ & |(d)| +ATR & |(e)| -ATR &  |(g)| +ATR &        &          & \\
            & |(4)| V    & |(5)| V    & |(6)| V     & |(c3)| C & |(c4)| C & |(7)| \LARGE$\#$ \\
            & |(f)| -low &           &             &         &          & \\
};
  \draw (d.south) -- (4.north);
  \draw (e.south) -- (5.north);
  \draw (g.south) -- (6.north);
  \draw foreach \x in {4, 5, 6} {(\x.south) -- (f.north)};
  \draw[black,->] (6) -- (c3);
  \draw[black,->] (c3) -- (c4);
  \draw[black,->] (c4) -- (7);
  \end{tikzpicture}
\end{exe}

\noindent The FSCs in (\ref{sgcons}) above all contain one or more
consonants in a successor relation with a vowel, a word boundary, or
both. Including the FSCs in (\ref{sgcons}) means that a sour grapes
vowel harmony pattern with word boundaries only on the segmental tier
can still be captured by FSCs, but it requires more FSCs than any other
pattern discussed in this paper.

\section{Discussion}\label{discussion}

A goal of generative phonology repeated throughout this paper is to
distinguish attested phonological patterns from unattested, but
logically possible patterns. This distiction is clearly made if we posit
a theory in which attested patterns are local and unattested patterns
are nonlocal. We can make this distinction based on locality by positing
that attested patterns must be describable by FSCs. One can use FSCs as
a definition of locality because they refer to elements within a
structure that are connected by either an ordering or association
relation. A phonological pattern is local if it can be described with
FSCs because it can be captured by referring to a subset of the elements
within structures and their connections. Jardine (2016) and Jardine
(2017) found that FSCs over two-tiered ARs are expressive enough to
capture a variety of attested tone patterns, which are thus local.
Similarly, this qualifying paper has demonstrated the suitability of
FSCs over multi-tiered ARs for capturing the attested vowel harmony
patterns in Akan, Finnish, and Turkish. The FSCs for these patterns
minimally consist of features on a tier connected to each other or a
morpheme boundary by the successor relation. Maximally, the FSCs for the
attested vowel harmony patterns examined here consist of vowels on a
segmental tier associated to two different feature tiers with the
successor relation connecting elements on one of the feature tiers.
Based on these results and the definition of locality provided above,
attested vowel harmony patterns are local.

However, FSCs over multi-tier ARs can also describe the unattested sour
grapes pattern. Lamont (2018) showed that FSCs over two-tiered ARs are
not expressive enough to capture an unattested sour grapes nasal harmony
pattern. However, the interaction between multiple feature tiers allows
FSCs over multi-tiered ARs to capture the so-called long distance
blocking effect of a sour grapes vowel harmony pattern. When word
boundaries are explicitly represented on the segmental or feature tiers,
the sour grapes FSCs follow the same conventions as the FSCs of attested
vowel harmony patterns: at least one vowel on the segmental tier is
associated to features on two different feature tiers and the successor
relation connects elements on only one feature tier. So, based on the
above definition sour grapes vowel harmony is also local.

But the problem remains of distinguishing attested vowel harmony
patterns from the unattested sour grapes vowel harmony pattern. Section
5 demonstrates that sour grapes vowel harmony can be captured by FSCs
over multi-tiered ARs when word boundaries are represented only on the
segmental tier or when word boundaries are represented on both the
segmental and feature tiers. The fact that the theory outlined in this
paper is not restrictive enough to exclude the unattested sour grapes
pattern may suggest that the multi-tiered ARs used here are too
expressive. Future work will investigate whether it is necessary to use
string-based representations rather than ARs or if a more expressive
class of grammars can capture attested vowel harmony patterns over
multi-tiered ARs while excluding unattested patterns like sour grapes.

\section{Conclusion}\label{conclusion}

This qualifying paper adopts a formal language theory approach to
determine the locality of vowel harmony patterns, but fails to
distinguish attested vowel harmony patterns from a logically possible
unattested sour grapes pattern. Using Jardine (2017)'s FSCs, attested
surface vowel harmony patterns are shown to be local. However, the
theory of well-formedness developed here is expressive enough to capture
attested vowel harmony patterns, but not restrictive enough to rule out
the unattested sour grapes pattern.

Unlike previous work on vowel harmony, this paper analyzes only surface
ARs to show that given a uniform theory of markedness constraints
attested vowel harmony patterns include those due to both spreading and
agreement. Despite being derived by different assimilation processes,
attested vowel harmony patterns can be considered as part of a single
set of local patterns because they can be captured by FSCs over
multi-tiered ARs.

Future work to be done on this topic will investigate the possibilities
of restricting the representation or increasing the expressive power of
the grammars that generate vowel harmony. One posibility is that it will
be necessary to use string-based representations rather than ARs to
represent vowel harmony patterns. Alternatively, a more expressive class
of grammars may be able to capture attested vowel harmony patterns over
multi-tiered ARs while excluding unattested patterns like sour grapes.

\newpage

\section{References}\label{references}

\setlength{\parindent}{-0.5in} \setlength{\leftskip}{0.5in}

\hypertarget{refs}{}
\hypertarget{ref-archangelipulleyblank1994}{}
Archangeli, D., \& Pulleyblank, D. (1994). \emph{Grounded phonology}
(Vol. 25). MIT Press.

\hypertarget{ref-bakovic2000}{}
Bakovic, E. (2000, January). \emph{Harmony, dominance, and control}
(PhD thesis). Rutgers University, New Brunswick, New Jersey.

\hypertarget{ref-chandleeheinz2018}{}
Chandlee, J., \& Heinz, J. (2018). Strict locality and phonological
maps. \emph{Linguistic Inquiry}, \emph{49}(1), 23--60.

\hypertarget{ref-chandleejardine2013}{}
Chandlee, J., \& Jardine, A. (2013). Learning phonological mappings by
learning strictly local functions. In \emph{Proceedings of the 2013
annual meeting on phonology}.

\hypertarget{ref-chandleeeyraudheinz2014}{}
Chandlee, J., Eyraud, R., \& Heinz, J. (2014). Learning strictly local
subsequential functions. In \emph{Transactions of the association for
computational linguistics} (Vol. 2, pp. 491--503).

\hypertarget{ref-Clements1976}{}
Clements, G. (1976). Vowel harmony in non-linear generative phonology:
An autosegmental model.

\hypertarget{ref-crothersshibatani1980}{}
Crothers, J., \& Shibatani, M. (1980). Issues in the description of
turkish vowel harmony. \emph{Issues in the Description of Turkish Vowel
Harmony}, 63--68.

\hypertarget{ref-delacy2011}{}
deLacy, P. (2011). Markedness and faithfulness constraints. In M. van
Oostendorp, C. Ewen, E. Hume, \& K. Rice (Eds.), (pp. 1--22). Blackwell.

\hypertarget{ref-Goldsmith1976}{}
Goldsmith, J. (1976). \emph{Autosegmental phonology} (PhD thesis).
Massachusetts Institute of Technology.

\hypertarget{ref-heinzidsardi2013}{}
Heinz, J., \& Idsardi, W. (2013). What complexity differences reveal
about domains in language. \emph{Topics in Cognitive Science},
\emph{5}(1), 111--131.

\hypertarget{ref-heinzetaltsl}{}
Heinz, J., Rawal, C., \& Tanner, H. G. (2011). Tier-based strictly local
constraints for phonology. In \emph{Proceedings of the 49th annual
meeting of the association for computational linguistics: Human language
technologies: Short papers} (Vol. 2). Association for Computational
Linguistics.

\hypertarget{ref-jardinediss}{}
Jardine, A. (2016). \emph{Locality and non-linear representations in
tonal phonology} (PhD thesis). University of Delaware.

\hypertarget{ref-jardinelocaltone}{}
Jardine, A. (2017). The local nature of tone association patterns.
\emph{Phonology}, \emph{34}(2), 385--405.

\hypertarget{ref-jardineexpressag}{}
Jardine, A. (2018). The expressivity of autosegmental grammars.

\hypertarget{ref-jardineheinz2015}{}
Jardine, A., \& Heinz, J. (2015a). A concatenation operation to derive
autosegmental graphs. In \emph{Proceedings of the 14th annual meeting on
the mathematics of language (mol 2015)} (pp. 139--151). Chicago, USA:
Association for Computational Linguistics.

\hypertarget{ref-jardineheinzcls}{}
Jardine, A., \& Heinz, J. (2015b). Markedess constraints are negative:
An autosegmental constraint definition language. In \emph{Proceedings of
the 51st annual meeting of the chicago linguistics society}.

\hypertarget{ref-lamont2018}{}
Lamont, A. (2018). ms. University of Massachusetts Amherst.

\hypertarget{ref-leben1973}{}
Leben, W. (1973). \emph{Suprasegmental phonology} (PhD thesis).
Massachusetts Institute of Technology.

\hypertarget{ref-mccarthyfg1988}{}
McCarthy, J. (1988). Feature geometry and dependency: A review.
\emph{Phonetica}, \emph{38}. Retrieved from
\url{http://scholarworks.umass.edu/linguist_faculty_pubs/38}

\hypertarget{ref-mccarthy2011}{}
McCarthy, J. (2011). Autosegmental spreading in optimality theory. In
\emph{Tones and features} (Clements Memorial Volume., Vol. 27).
Retrieved from
\url{https://scholarworks.umass.edu/linguist_faculty_pubs/27}

\hypertarget{ref-Nevins2010}{}
Nevins, A. (2010). \emph{Locality in vowel harmony}. \emph{Linguistic
Inquiry Monographs} (Vol. 55). MIT Press.

\hypertarget{ref-odden1994}{}
Odden, D. (1994). Adjacency parameters in phonology. \emph{Language},
\emph{70}(2), 289--330.

\hypertarget{ref-padgett1995}{}
Padgett, J. (1995). Feature classes. In J. Beckman, S. Urbanczyk, \& L.
Walsh (Eds.), \emph{Papers in optimality theory} (Vol. 18, pp.
385--420).

\hypertarget{ref-padgett2002}{}
Padgett, J. (2002). Feature classes in phonology. \emph{Language},
\emph{78}(1), 81--110. Retrieved from
\url{http://www.jstor.org/stable/3086646}

\hypertarget{ref-princesmolensky1993}{}
Prince, A., \& Smolensky, P. (1993). \emph{Optimality theory: Constraint
interaction in generative grammar} (No. 2). Rutgers University Center
for Cognitive Science.

\hypertarget{ref-princesmolensky2004}{}
Prince, A., \& Smolensky, P. (2004). \emph{Optimality theory: Constraint
interaction in generative grammar}. Blackwell.

\hypertarget{ref-ringenheinamaki1999}{}
Ringen, C., \& Heinamaki, O. (1999). Variation in finnish vowel harmony:
An ot account. \emph{Natural Languge and Linguistic Theory}, \emph{17},
303--337.

\hypertarget{ref-ringenvago1998}{}
Ringen, C., \& Vago, R. (1998). Hungarian vowel harmony in optimality.
\emph{Phonology}, \emph{15}, 393--416.

\hypertarget{ref-rogerspullum2011}{}
Rogers, J., \& Pullum, G. (2011). Aural pattern recognition experiments
and the subregular hierarchy. \emph{Journal of Logic, Language, and
Information}, \emph{20}, 329--342.

\hypertarget{ref-rogersetal2013}{}
Rogers, J., Heinz, J., Fero, M., Hurst, J., Lambert, D., \& Wibel, S.
(2013). Cognitive and sub-regular complexity. \emph{Formal Grammar},
90--108.

\hypertarget{ref-rosewalker2011}{}
Rose, S., \& Walker, R. (2011). Harmony systems. In J. Goldsmith, J.
Riggle, \& A. Yu (Eds.), \emph{The handbook of phonological theory} (pp.
240--290). Blackwell.

\hypertarget{ref-sagey1986}{}
Sagey, E. (1986). \emph{The representation of features and relations in
non-linear phonology} (PhD thesis). Massachusetts Institute of
Technology.

\hypertarget{ref-vdHulst2017}{}
van der Hulst, H. (2017). A representational account of vowel harmony in
terms of variable elements and licensing. In \emph{Approaches to
hungarian} (Vol. 15). John Benjamins Publishing Company.

\hypertarget{ref-vdHulstSmith1986}{}
van der Hulst, H., \& Smith, N. (1986). On neutral vowels. In \emph{The
phonological representation of suprasegmentals} (pp. 233--281).

\hypertarget{ref-valimaablum1986}{}
Välimaa-Blum, R. (1986). Finnish vowel harmony as a prescriptive and
descriptive rule: An autosegmental account. In F. Marshall (Ed.),
\emph{Proceedings of the third eastern states conference on
linguistics}. University of Pittsburgh.

\hypertarget{ref-walker2010}{}
Walker, R. (2010). Nonmyopic harmony and the nature of derivations.
\emph{Linguistic Inquiry}, \emph{41}(1), 169--179.

\hypertarget{ref-walker2014}{}
Walker, R. (2014). Surface correspondence and discrete harmony triggers.
In \emph{Proceedings of the annual meetings on phonology}.


\end{document}
