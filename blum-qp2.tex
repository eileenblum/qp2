\documentclass[floatsintext,man]{apa6}

\usepackage{amssymb,amsmath}
\usepackage{ifxetex,ifluatex}
\usepackage{fixltx2e} % provides \textsubscript
\ifnum 0\ifxetex 1\fi\ifluatex 1\fi=0 % if pdftex
  \usepackage[T1]{fontenc}
  \usepackage[utf8]{inputenc}
\else % if luatex or xelatex
  \ifxetex
    \usepackage{mathspec}
    \usepackage{xltxtra,xunicode}
  \else
    \usepackage{fontspec}
  \fi
  \defaultfontfeatures{Mapping=tex-text,Scale=MatchLowercase}
  \newcommand{\euro}{€}
\fi
% use upquote if available, for straight quotes in verbatim environments
\IfFileExists{upquote.sty}{\usepackage{upquote}}{}
% use microtype if available
\IfFileExists{microtype.sty}{\usepackage{microtype}}{}

% Table formatting
\usepackage{longtable, booktabs}
\usepackage{lscape}
% \usepackage[counterclockwise]{rotating}   % Landscape page setup for large tables
\usepackage{multirow}		% Table styling
\usepackage{tabularx}		% Control Column width
\usepackage[flushleft]{threeparttable}	% Allows for three part tables with a specified notes section
\usepackage{threeparttablex}            % Lets threeparttable work with longtable

% Create new environments so endfloat can handle them
% \newenvironment{ltable}
%   {\begin{landscape}\begin{center}\begin{threeparttable}}
%   {\end{threeparttable}\end{center}\end{landscape}}

\newenvironment{lltable}
  {\begin{landscape}\begin{center}\begin{ThreePartTable}}
  {\end{ThreePartTable}\end{center}\end{landscape}}




% The following enables adjusting longtable caption width to table width
% Solution found at http://golatex.de/longtable-mit-caption-so-breit-wie-die-tabelle-t15767.html
\makeatletter
\newcommand\LastLTentrywidth{1em}
\newlength\longtablewidth
\setlength{\longtablewidth}{1in}
\newcommand\getlongtablewidth{%
 \begingroup
  \ifcsname LT@\roman{LT@tables}\endcsname
  \global\longtablewidth=0pt
  \renewcommand\LT@entry[2]{\global\advance\longtablewidth by ##2\relax\gdef\LastLTentrywidth{##2}}%
  \@nameuse{LT@\roman{LT@tables}}%
  \fi
\endgroup}


\ifxetex
  \usepackage[setpagesize=false, % page size defined by xetex
              unicode=false, % unicode breaks when used with xetex
              xetex]{hyperref}
\else
  \usepackage[unicode=true]{hyperref}
\fi
\hypersetup{breaklinks=true,
            pdfauthor={},
            pdftitle={On the locality of vowel harmony over multi-tiered autosegmental representations},
            colorlinks=true,
            citecolor=blue,
            urlcolor=blue,
            linkcolor=black,
            pdfborder={0 0 0}}
\urlstyle{same}  % don't use monospace font for urls

\setlength{\parindent}{0pt}
%\setlength{\parskip}{0pt plus 0pt minus 0pt}

\setlength{\emergencystretch}{3em}  % prevent overfull lines


% Manuscript styling
\captionsetup{font=singlespacing,justification=justified}
\usepackage{csquotes}
\usepackage{upgreek}



\usepackage{tikz} % Variable definition to generate author note

% fix for \tightlist problem in pandoc 1.14
\providecommand{\tightlist}{%
  \setlength{\itemsep}{0pt}\setlength{\parskip}{0pt}}

% Essential manuscript parts
  \title{On the locality of vowel harmony over multi-tiered autosegmental
representations}

  \shorttitle{Locality of vh over multi-tiered ARs}


  \author{Eileen Blum\textsuperscript{1}}

  % \def\affdep{{""}}%
  % \def\affcity{{""}}%

  \affiliation{
    \vspace{0.5cm}
          \textsuperscript{1} Rutgers University  }

  \authornote{
    Correspondence concerning this article should be addressed to Eileen
    Blum, 18 Seminary Place, New Brunswick, NJ 08901. E-mail:
    \href{mailto:eileen.blum@rutgers.edu}{\nolinkurl{eileen.blum@rutgers.edu}}
  }


  \keywords{keywords \\

    \indent Word count: X
  }




  \usepackage{tipa}
  \usepackage{gb4e}
  \noautomath
  \usepackage{tikz}
  \usetikzlibrary{matrix}
  \tikzset{marked/.style={draw=none, fill=none}}

\usepackage{amsthm}
\newtheorem{theorem}{Theorem}
\newtheorem{lemma}{Lemma}
\theoremstyle{definition}
\newtheorem{definition}{Definition}
\newtheorem{corollary}{Corollary}
\newtheorem{proposition}{Proposition}
\theoremstyle{definition}
\newtheorem{example}{Example}
\theoremstyle{definition}
\newtheorem{exercise}{Exercise}
\theoremstyle{remark}
\newtheorem*{remark}{Remark}
\newtheorem*{solution}{Solution}
\begin{document}

\maketitle

\setcounter{secnumdepth}{0}



\section{Introduction}\label{introduction}

This qualifying paper aims to investigate the locality of vowel harmony
patterns over autosegmental representations (ARs) using Jardine (2017)'s
forbidden substructure constraints (FSCs) over ARs. The investigation
will provide a well-defined, computationally motivated theory of
well-formedness in vowel harmony. Jardine (2017) developed a theory of
tonal well-formedness and determined that tone patterns are
fundamentally local over two-tiered ARs. Investigating the locality of
vowel harmony patterns will determine the expressive power of
multi-tiered ARs and allows for a theory of well-formedness that makes
accurate typological predictions.

A goal shared by all of generative phonology is to distinguish attested
patterns from logically possible, but unattested ones. A theory of
well-formedness in vowel harmony that accomplishes this goal must be
both expressive enough to explain the attested typology of vowel harmony
patterns and restrictive enough to exclude the logically possible
unattested vowel harmony patterns. This qualifying paper will adopt a
formal language theory approach that provides explicit ways of
determining the expressivity and restrictiveness of phonological
patterns.

\subsection{The formal language theory
approach}\label{the-formal-language-theory-approach}

The goal of distinguishing attested phonological patterns from possible
unattested patterns is currently being investigated using formal
language theory to determine the expressive power required to compute
phonological patterns in general. The Chomsky hierarchy, in
(\ref{chomsky.hierarchy}), classifies stringsets in terms of the
relative expressivity of the grammars needed to generate them. Each
class that is lower on the hierarchy is also a proper subset of the
class above it.

\begin{exe}
\ex{The Chomsky Hierarchy:}\label{chomsky.hierarchy} 
Finite $\subsetneq$ Regular $\subsetneq$ Context-Free $\subsetneq$ Context-Sensitive $\subsetneq$ Computably Enumerable
\end{exe}

A significant body of work in computational phonology shows that
phonological generalizations are properly contained within the regular
class of stringsets (Heinz \& Idsardi, 2013). Recent work has further
established a subregular hierarchy of stringset classes, i.e.~star-free
(SF) and weaker classes (Heinz, Rawal, \& Tanner, 2011, Rogers and
Pullum (2011); Rogers et al., 2013). A generative phonlogical theory
must be expressive enough to predict the regular patterns and
restrictive enough to rule out patterns that fall into a larger class,
such as context-free. The classifications of stringsets and ARs in this
manner are not directly comparable, but Jardine (2018; following Jardine
and Heinz, 2015) provides a method for comparing the expressivity of the
grammars that generate them. Jardine (2018) establishes a sub-SF class
of \enquote{forbidden k-factor grammars} over ARs,
ASL\textsuperscript{g}, that is expressive enough to capture a range of
attested tone patterns (ASL\textsuperscript{g\textsubscript{T}}). The
ASL\textsuperscript{g\textsubscript{T}} class includes patterns that
would also fall within three different subregular stringset classes:
strictly local (SL), tier-based strictly local (TSL), and strictly
piecewise (SP). The goal of this qualifying paper is to determine
whether or not the typology of vowel harmony patterns must be captured
by a class of grammars, using ARs with more than two tiers, that is more
expressive than the ASL\textsuperscript{g\textsubscript{T}} class.

Patterns represented with multi-tiered ARs demonstrate whether or not
enriching the representation necessarily increases the expressivity of a
grammar. Representations of vowel harmony refer to subsegmental
features, which will be represented using multiple featural tiers, such
that each feature occupies a separate tier that is associated to a vowel
on the melody tier (following Clements, 1976; McCarthy, 1988). Such ARs
include at least one additional tier compared with the ARs of tone
patterns, which utilize only two tiers (Jardine, 2016, 2017, 2018). This
qualifying paper will determine whether or not enriching ARs in this way
increases the expressivity of a grammar such that it falls outside of
the previously established ASL\textsuperscript{g\textsubscript{T}}
class. Three aspects of multi-tiered ARs of vowel harmony are
investigated: the complexity of vowel harmony generalizations that
include domain information in Turkish and Finnish, the locality of an
asymmetry between harmony triggers and undergoers in Baiyina Oroquen
(Walker, 2014a), and whether or not multi-tiered ARs predict the
generation of an unattested pattern: \enquote{sour grapes} (McCarthy,
2011; Padgett, 1995; Walker, 2010). Each of these investigations will
provide additional evidence of the expressive power needed for a grammar
to generate vowel harmony patterns.

Vowel harmony can be viewed either as an input-output map or as a
phonotactic \enquote{cooccurrence restriction upon the vowels that may
occur in a word} (Clements, 1976). A hierarchy that classifies sets of
ARs, based on the Chomsky and related subregular hierarchies, differs
significantly from a parallel hierarchy for sets of pairs of ARs, such
as in a transformation (or map) from underlying to surface form.
Previous analyses use ARs to describe vowel harmony patterns as the
spreading of a vowel feature from one vowel throughout the word until it
is blocked (Clements, 1976; Goldsmith, 1976; McCarthy, 1988; Padgett,
2002; Sagey, 1986; vanderHulst, 2017; Walker, 2010, 2014b). Clements
(1976)'s well-formedness condition motivates feature spreading in order
to ensure that all elements on one tier of an AR are connected via an
association relation to some element on another tier of the same AR. The
result is an AR in which all elements on one tier are associated to some
element on another tier. A majority of scholars have thus viewed vowel
harmony as mapping an input with a vowel feature associated to one vowel
onto an output where that same feature is associated to multiple vowels.

This qualifying paper constitutes the first in-depth formal language
theoretic study of vowel harmony as a phonotactic restriction rather
than an input-output map. ARs within the Chomsky-based hierarchy cannot
be compared with sets of pairs of ARs within a separate hierarchy. In
order to compare vowel harmony with other patterns classified within the
subregular hierarchy of ARs this qualifying paper will be taking a
slightly different approach than has been taken before by evaluating
only the restrictions on output substructures. While vowel harmony has
been considered a derivational process, this paper will aim to determine
the locality of only the surface restrictions on vowel harmony patterns
over ARs. The harmonizing ARs that will be examined contain at least one
feature that is associated to more than one vowel, as it would be on the
surface. Ignoring input structures in this way allows for the
classification of vowel harmony within the sub-SF hierarchy of patterns,
which allows for the comparison of vowel harmony with other phonological
patterns that have been classified on the same hierarchy, such as tone
in Jardine (2018).

\subsection{Motivating Autosegmental Representations
(ARs)}\label{motivating-autosegmental-representations-ars}

This qualifying paper will determine the locality of surface
restrictions on vowel harmony patterns over multi-tiered ARs by
investigating whether they can be captured using Jardine (2017)'s
\enquote{forbidden substructure constraints}(FSCs). FSCs are defined as
surface markedness constraints (OT; Prince \& Smolensky, 1993, 2004),
\enquote{which ban pieces of autosegmental representations} (Jardine,
2017, p. 1). FSCs serve as a type of phonotactic restriction such that
\enquote{well-formedness is based on contiguous structures of a specific
size} (Jardine, 2017, p. 3). One can use FSCs as a definition of
locality because they refer to elements within a structure that are
connected by either an ordering or association relation. A phonological
pattern is thus local if it can be described with FSCs because it can be
captured by referring to a subset of the elements within structures and
their connections. Jardine (2017) uses FSCs to show that attested tone
patterns are local in this way.

\subsubsection{Multi-tiered ARs}\label{multi-tiered-ars}

Autosegmental representations (ARs) of tonal patterns generally consist
of two tiers: the TBU and melody tiers (Goldsmith, 1976; Jardine, 2016,
2017), but an open question that remains is: From a formal perspective,
what is the range of patterns that can be represented using more than
two autosegmental tiers? This paper investigates the expressive power
needed to represent one such pattern. Vowel harmony patterns refer to
subsegmental features, which will be represented using multiple featural
tiers; each feature occupies a separate tier that is associated to a
vowel on the melody tier (following Clements, 1976; McCarthy, 1988). For
example, assuming binary features, vowel features like {[}\(\pm\)
back{]}, {[}\(\pm\) high{]}, etc. are represented on separate tiers and
associated to a vowel on the melody tier, as in (\ref{fsc.ex}).

\begin{exe}
\ex \label{fsc.ex}
  \begin{tikzpicture}[baseline=(current bounding box.north)]
  \matrix [matrix of nodes,
          row sep=1em, 
          nodes={text height=1em, text depth=0.5em}] {
|(h)| $\pm$ high\\
|(v)| V         \\
|(b)| $\pm$ back\\
  };
   \draw (h.south) -- (v.north);
   \draw (v.south) -- (b.north);
  \end{tikzpicture}
\end{exe}

The goal for this project is to extend the work of Jardine (2017) to
determine whether vowel harmony patterns are local over ARs with more
than two tiers, as in (\ref{fsc.ex}). This qualifying paper evaluates
whether or not the restrictions on attested vowel harmony patterns can
be captured using FSCs.

\subsubsection{Representational
assumptions}\label{representational-assumptions}

Use of ARs requires discussion of at least some of the basic
representational assumptions held throughout this paper. The basic
assumptions include Clements (1976)'s Well-Formedness Condition, which
includes stipulations of \emph{Full Specification} (FS), the \emph{No
Crossing Constraint} (NCC) (Goldsmith, 1976; Sagey, 1986), and the
\emph{Obligatory Contour Principle} (OCP) (Leben, 1973). Examples of
structures that violate each of these assumptions are shown in
(\ref{akan.ex2})-(\ref{akan.ex4}) below.

\begin{exe}
\ex \label{akan.ex2} Violates FS
  \begin{tikzpicture}[baseline=(current bounding box.north)]
  \matrix [matrix of nodes, row sep=2.5ex, column sep=2.25ex, nodes={text height=1em, text depth=0.5em}] 
  {
* & |(b)| -ATR & \\
  & |(d)| V    & |(e)| V \\
  & |(f)| -low & \\
  };
  \draw (b.south) -- (d.north);
  \draw (d.south) -- (f.north);
  \draw[thick,black,->] (d) -- (e);
  \draw foreach \x in {d, e} {(\x.south) -- (f.north)};
  \end{tikzpicture}
\end{exe}

\begin{exe}
\ex \label{akan.ex3} Violates NCC
  \begin{tikzpicture}[baseline=(current bounding box.north)]
  \matrix [matrix of nodes, row sep=2.5ex, column sep=2.25ex, nodes={text height=1em, text depth=0.5em}] 
  {
* & |(a)| +ATR & |(b)| -ATR \\
  & |(c)| V    & |(d)| V \\
  & |(e)| -low &  \\
  };
  \draw (a.south) -- (d.north);
  \draw (b.south) -- (c.north);
  \draw[thick,black,->] (a) -- (b);
  \draw[thick,black,->] (c) -- (d);
  \draw foreach \x in {c, d} {(\x.south) -- (e.north)};
  \end{tikzpicture}
\end{exe}

\begin{exe}
\ex \label{akan.ex4} Violates OCP
  \begin{tikzpicture}[baseline=(current bounding box.north)]
  \matrix [matrix of nodes, row sep=2.5ex, column sep=2.25ex, nodes={text height=1em, text depth=0.5em}] 
  {
* & |(a)| -ATR & |(b)| -ATR \\
  & |(c)| V    & |(d)| V \\
  & |(e)| -low & |(f)| -low \\
  };
  \draw (a.south) -- (c.north);
  \draw (b.south) -- (d.north);
  \draw (c.south) -- (e.north);
  \draw (d.south) -- (f.north);
  \draw[thick,black,->] (a) -- (b);
  \draw[thick,black,->] (c) -- (d);
  \draw[thick,black,->] (e) -- (f);
  \end{tikzpicture}
\end{exe}

First, FS means that each featural element must be associated with at
least one vowel on the melody tier and each vowel on the melody tier
must be associated with at least one element on a featural tier. FS
crucially allows vowels to be associated with multiple featural tiers as
is necessary for each vowel feature to occupy its own tier. The
hypothetical representation in (\ref{akan.ex2}) straighforwardly
violates FS because there is a vowel that is not associated to any
feature on the ATR tier. While both vowels are associated to a single
-low feature, the second vowel is not associated to any feature on the
ATR tier. Since vowel harmony patterns will be analyzed, it will be
assumed that consonants cannot be associated to vowel features and that
FS and vowel harmony in general ignore consonantal elements on the
melody tier.

Second, the NCC states that association lines between the melody tier
and a feature tier never cross. Odden (1994) adds that the NCC can only
evaluate the association between the melody and one featural tier at a
time. The representation in (\ref{akan.ex3}) violates the NCC because
+ATR precedes -ATR, but is associated to a vowel that is preceded by a
vowel associated to -ATR; this configuration creates visually crossed
association lines.

A notable effect of FS along with the NCC is that they prevent what have
been called gapped structures (Archangeli \& Pulleyblank, 1994; Ringen
\& Vago, 1998). A gapped structure is one in which a feature appears to
have skipped over a vowel that it could potentially be associated to. FS
would prevent gapped structures in which the \enquote{skipped} vowel is
not associated to anything on that particular feature's tier. The NCC
would prevent gapped structures in which the surrounding two vowels are
associated to the same feature and the \enquote{skipped} vowel is
associated to a different feature on the same tier.

Lastly, the OCP stipulates that adjacent featural elements must be
distinct. The representation in (\ref{akan.ex4}) violates the OCP
because on both the ATR and low feature tiers there are two identical
adjacent features, -ATR and -low respectively. The OCP in conjunction
with FS results in representations where multiple adjacent vowels are
associated to a single feature rather than having multiple adjacent
iterations of the same feature each associated to a single vowel. An
example representation of an Akan word that satisfies all of the AR
properties discussed here is shown in (\ref{akan.ex}).

\begin{exe}
\ex \label{akan.ex} Satisfies FS, NCC, and OCP
  \begin{tikzpicture}[baseline=(current bounding box.north)]
  \matrix [matrix of nodes, row sep=2.5ex, column sep=2.25ex, nodes={text height=1em, text depth=0.5em}] 
  {
  &            & |(b)| -ATR & \\
  & |(c)| t    & |(d)| i    & |(e)| e \\
  &            & |(f)| -low & \\
  };
  \draw foreach \x in {d, e} {(b.south) -- (\x.north)};
  \draw foreach \x in {d, e} {(\x.south) -- (f.north)};
  \draw[thick,black,->] (c) -- (d);
  \draw[thick,black,->] (d) -- (e);
  \end{tikzpicture}
\end{exe}

Again, the initial consonant cannot be associated to a vowel feature so,
while it is ordered with respect to the vowels, FS does not require the
consonant to be associated to any element on either feature tier. The AR
of \emph{tie} satisfies FS because each vowel is associated to a feature
on each of the featural tiers and all features are associated to at
lease on vowel. The AR of \emph{tie} also satisfies both the NCC and the
OCP because there is only one of each feature. The features are
represented on separate tiers so association lines cannot cross and
there is nothing else on those tiers that could violate the OCP. In
addition, (\ref{akan.ex2})-(\ref{akan.ex}) illustrate that, unlike the
usual notation, this paper will be adding a representation of the
successor ordering relation on each tier using arrows.

\section{A Generic case of vowel
harmony}\label{a-generic-case-of-vowel-harmony}

This paper begins with an example vowel harmony pattern that can be
captured with a FSC: ATR harmony in Akan (Clements, 1976). The Akan
vowel inventory, in Table \ref{akan_vowels}, consists of ten vowels with
two main featural distinctions: \(\pm\) ATR and \(\pm\) low. There are
two +low vowels, {[}\textipa{3}{]} and {[}a{]}, +ATR and -ATR,
respectively. All other vowels are considered -low and distinguished by
ATR such that the +ATR vowels are {[}i, e, u, o{]} and the -ATR vowels
are {[}\textipa{I, E, U, O}{]}.

\begin{table}
  \caption{Akan Vowels}
  \begin{tabular}{c|c|c}
       & +ATR        & -ATR       \\\hline\hline
  -low & i           & \textipa{I}\\ \cline{2-3}
       & u           & \textipa{U}\\ \cline{2-3}
       & e           & \textipa{E}\\ \cline{2-3}
       & o           & \textipa{O}\\\hline
  +low & \textipa{3} & a\\\hline
  \end{tabular}
  \label{akan_vowels}
\end{table}

The harmony generalization is that if a word contains a sequence of -low
vowels, then those vowels will also share the same ATR feature
(Clements, 1976). For example, the words in (\ref{akan_-low}) contain
only -low vowels, which are also all either +ATR or -ATR.

\begin{exe}
\ex{-low vowels share an ATR feature value}\label{akan_-low}
\begin{xlist}
  \ex tie `listen'                                             \\
  \ex obejii `he came and removed it' \\
  \ex \textipa{O}b\textipa{E}j\textipa{E}\textipa{I} `he came and did it' \\
  \ex wubenum\textraiseglotstop   `you will suck it'             \\
  \ex w\textipa{U}b\textipa{E}n\textipa{U}m\textraiseglotstop   `you will drink it' \\
  \end{xlist}
\end{exe}

The surface requirement that adjacent -low vowels share the same ATR
feature can also be written as a FSC, which forbids two adjacent vowels
associated to the same -low feature from being associated to different
ATR features, as in (\ref{fsc.akana}). The ordering relation on the ATR
tier in (\ref{fsc.akana}) is omitted because the + or - values of the
two ATR features are irrelevant for this constraint, as long as they
differ. The ordering relation on the melody tier of this FSC is also
omitted and the reason will be made clear by the example in
(\ref{akan.obeijii}).

\begin{exe}
\ex \label{fsc.akana}
  \begin{tikzpicture}[baseline=(current bounding box.north)]
  \matrix [matrix of nodes, row sep=2.5ex, column sep=2.25ex, nodes={text height=1em, text depth=0.5em}] 
  {
* & |(a)| +ATR & |(f)| -ATR \\
  & |(c)| V    & |(h)| V \\
  & |(e)| -low &  \\
  };
  \draw (a.south) -- (c.north);
  \draw (f.south) -- (h.north);
  \draw foreach \x in {c, h} {(\x.south) -- (e.north)};
  \end{tikzpicture}
\end{exe}

\begin{exe}
\ex{[obejii] `he came and removed it'}\label{akan.obeijii}
  \begin{tikzpicture}[baseline=(current bounding box.north)]
  \matrix [matrix of nodes, row sep=4ex, column sep=2.25ex, nodes={text height=1em, text depth=0.5em}] 
  {
(a) & |(a)| +ATR  &         &         &         &         &         & (b) * & |[red](i)|  \textbf{+ATR}  &         &         &         & |[red](j)| \textbf{-ATR}    & \\
    & |(b)| o     & |(c)| b & |(d)| e & |(e)| j & |(f)| i & |(g)| i &       & |(k)| o     & |(l)| b & |[red](m)| \textbf{e} & |(n)| j & |[red](o)| \textbf{\textipa{I}} & |(p)| \textipa{I} \\
    & |(h)| -low  &         &         &         &         &         &       & |[red](q)| \textbf{-low} & \\
  };
  \draw foreach \x in {b, d, f, g} {(a.south) -- (\x.north)};
  \draw foreach \x in {b, d, f, g} {(\x.south) -- (h.north)};
  \draw (i.south) -- (k.north);
  \draw[red,thick] (i.south) -- (m.north);
  \draw[red,thick] (j.south) -- (o.north);
  \draw (j.south) -- (p.north);
  \draw foreach \x in {k, p} {(\x.south) -- (q.north)};
  \draw[red,thick] foreach \x in {m, o} {(\x.south) -- (q.north)};
  \draw[black,->] (b) -- (c);
  \draw[black,->] (c) -- (d);
  \draw[black,->] (d) -- (e);
  \draw[black,->] (e) -- (f);
  \draw[black,->] (f) -- (g);
  \draw[black,->] (i) -- (j);
  \draw[black,->] (k) -- (l);
  \draw[black,->] (l) -- (m);
  \draw[black,->] (m) -- (n);
  \draw[black,->] (n) -- (o);
  \draw[black,->] (o) -- (p);
  \end{tikzpicture}
\end{exe}

The AR for the grammatical Akan word {[}obeijii{]} \enquote{he came and
removed it} is shown in (\ref{akan.obeijii}a). Here a single +ATR and a
single -low feature are each associated to each vowel within the word,
demonstrating full ATR and low harmony. On the other hand, the
hypothetical Akan word, {[}obej\textipa{II}{]}, represented in
(\ref{akan.obeijii}b) is ungrammatical because it demonstrates full -low
harmony, but does not demonstrate full ATR harmony; so, the AR in
(\ref{akan.obeijii}b) contains the forbidden structure of
(\ref{fsc.akana}), shown in bold and red.

However, in traditional vowel harmony terms the presence of a +low vowel
blocks the rightward spread of ATR, some examples are shown in
(\ref{akan_+low}). Translating this to the static surface
representations assumed here, two -low vowels must be associated to the
same ATR feature, but if a +low vowel intervenes they can be associated
to different ATR features. The representation of (\ref{akan_+low}a)
exemplifies this pattern and is shown in (\ref{ex.akan+low}).

\begin{exe}
  \ex{Vowels on either side of +low can have different ATR features}\label{akan_+low} 
  \begin{xlist}
    \ex p\textipa{I}r\textipa{3}ko  `pig'                                        \\
    \ex obisa\textipa{I} `he asked'                                              \\
    \ex m\textipa{I}k\textipa{O}k\textipa{3}ri  `I go and weight it'   \\
    \ex okog\textsuperscript{w}\textsuperscript{j}ar\textipa{I}\textraiseglotstop `he goes and washes' \\
  \end{xlist}
\end{exe}

\begin{exe}
\ex{[p\textipa{I}r\textipa{3}ko] `pig'}\label{ex.akan+low} 
\begin{tikzpicture}[baseline=(current bounding box.north)]
  \matrix [matrix of nodes, 
          row sep=2.5ex, column sep=2.25ex,
          nodes={text height=1em, text depth=0.5em}] 
  {
        & |(a)| -ATR        &         &                   &         & |(b)| +ATR\\
|(c)| p & |(d)| \textipa{I} & |(e)| r & |(f)| \textipa{3} & |(g)| k & |(h)| o\\
        & |(i)| -low        &         & |(j)| +low        &         & |(k)| -low\\
  };
  \draw foreach \x in {d, f} {(a.south) -- (\x.north)};
  \draw (b.south) -- (h.north);
  \draw (d.south) -- (i.north);
  \draw (f.south) -- (j.north);
  \draw (h.south) -- (k.north);
  \draw[black,->] (a) -- (b);
  \draw[black,->] (c) -- (d);
  \draw[black,->] (d) -- (e);
  \draw[black,->] (e) -- (f);
  \draw[black,->] (f) -- (g);
  \draw[black,->] (g) -- (h);
  \draw[black,->] (i) -- (j);
  \draw[black,->] (j) -- (k);
  \end{tikzpicture}
\end{exe}

Crucially, the AR in (\ref{ex.akan+low}) does not contain the FSC from
(\ref{fsc.akana}). While the AR for {[}p\textipa{Ir3}ko{]} \enquote{pig}
does contain two vowels associated to a -low feature and two different
ATR features, the presence of a blocking {[}\textipa{3}{]} vowel
associated to a +low feature causes the surrounding vowels to be
associated to two separate -low features in order to satisfy FS and the
NCC. Because the forbidden structure is not present
{[}p\textipa{Ir3}ko{]} \enquote{pig} is grammatical.

In summary, the basic vowel harmony pattern of Akan, can be captured
using the FSC in (\ref{fsc.akana}), which does not refer to adjacency on
any tier. Akan vowel harmony could thus be considered local because the
FSC that captures the pattern need only refer to the associations
between vowels and features. The next sections outline similar analyses
of a variety of other vowel harmony patterns in order to determine
whether or not there are any patterns that FSCs cannot capture and would
thus fall outside of the ASL\textsuperscript{gT} class.

\section{Domain-sensitive vowel
harmony}\label{domain-sensitive-vowel-harmony}

\subsection{Finnish}\label{finnish}

Native Finnish words demonstrate backness harmony with four transparent
front vowels. The Finnish vowel inventory in Table \ref{finnish_vowels}
consists of 16 vowels with contrastive length and two main featural
distinctions that will be the focus of this analysis: \(\pm\) back and
\(\pm\) neutral (Ringen \& Heinamaki, 1999; Välimaa-Blum, 1986). The
four vowels transparent to backness harmony, \textipa{[i, i:, e, e:]},
are all +neutral and -back. All other Finnish vowels are considered
-neutral and distinguished by backness such that the +back vowels are
\textipa{[u, u:, o, o:, A, A:]} and the -neutral, -back vowels are
\textipa{[y, y:, \o, \o:, \ae, \ae:]}.

\begin{table}
  \caption{Finnish Vowels}
  \begin{tabular}{c|c|c}
  \multicolumn{2}{l|}{-back}          & +back\\\hline\hline
  \textipa{i, i:} & \textipa{y, y:}   & \textipa{u, u:}\\ \cline{1-3}
  \textipa{e, e:} & \textipa{\o, \o:}  & \textipa{o, o:}\\ \cline{1-3}
                  & \textipa{\ae, \ae:} & \textipa{A, A:}\\ \cline{1-3}\hline\hline
  +neutral        & \multicolumn{2}{|l}{-neutral} \\
  \end{tabular}
  \label{finnish_vowels}
\end{table}

The Finnish harmony generalization is that all the -neutral vowels in a
root will share the same back feature with each other and -neutral
suffix vowels will share the same back feature with the root-final
-neutral vowel (Ringen \& Heinamaki, 1999; Välimaa-Blum, 1986). For
example, the words in (\ref{finnish_-neutral}) contain only -neutral
vowels, which are also all either +back or -back. A hyphen represents a
morpheme boundary.

\begin{exe}
\ex{-neutral vowels share a back feature value}\label{finnish_-neutral}
\begin{xlist}
  \ex \textipa{p\o yt\ae} `table' \\
  \ex \textipa{k\ae n-t\ae:} `turn' \\
  \ex \textipa{tyk\ae-t\ae} `like' \\
  \ex \textipa{poutA} `fine weather' \\
  \ex \textipa{mur-tA:} `break' \\
  \ex \textipa{kokA-tA} `cook' \\
  \end{xlist}
\end{exe}

The surface requirement that -neutral vowels share the same ATR feature
can also be written as a FSC, which forbids two vowels associated to the
same -neutral feature from being associated to different back features,
as in (\ref{fsc.finnish}). The general Finnish FSC does not include
morpheme boundaries because the same constraint applies within roots and
across morpheme boundaries. The ordering relation on the back tier in
(\ref{fsc.finnish}) is omitted because the + or - values of the two back
features are irrelevant for this constraint, as long as they differ. The
ordering relation on the melody tier of this FSC is also omitted because
the vowels can have consonants between them, as in
(\ref{finnish.poytana}).

\begin{exe}
\ex \label{fsc.finnish}
  \begin{tikzpicture}[baseline=(current bounding box.north)]
  \matrix [matrix of nodes, row sep=2.5ex, column sep=2.25ex, nodes={text height=1em, text depth=0.5em}] 
  {
* & |(a)| +back & |(f)| -back \\
  & |(c)| V    & |(h)| V \\
  & |(e)| -neutral &  \\
  };
  \draw (a.south) -- (c.north);
  \draw (f.south) -- (h.north);
  \draw foreach \x in {c, h} {(\x.south) -- (e.north)};
  \end{tikzpicture}
\end{exe}

\begin{exe}
\ex{\textipa{[p\o yt\ae-n\ae]} `table (essive)'}\label{finnish.poytana}
  \begin{tikzpicture}[baseline=(current bounding box.north)]
  \matrix [matrix of nodes, row sep=4ex, column sep=2.25ex, nodes={text height=1em, text depth=0.5em}] 
  {
(a) &          & |(a)| -back    &         &         &           &         & \\ 
    & |(b)| p  & |(c)| \o       & |(d)| y & |(e)| t & |(f)| \ae & |(g)| n & |(r)| \ae \\
    &          & |(h)| -neutral &         &         &           &         &  \\          
(b) * &         & |[red](i)| \textbf{-back} &         &         &                                            &         & |[red](j)| \textbf{+back} \\
      & |(k)| p & |(l)| \o                  & |(m)| y & |(n)| t & |[red](o)| \textbf{\ae}                    & |(p)| n & |[red](s)| \textbf{\textipa{A}} \\
      &         & |[red](q)| \textbf{-neutral} &                       &         &                              &         & \\
  };
  \draw foreach \x in {c, d, f, r} {(a.south) -- (\x.north)};
  \draw foreach \x in {c, d, f, r} {(\x.south) -- (h.north)};
  \draw foreach \x in {l, m} {(i.south) -- (\x.north)};
  \draw[red,thick] (i.south) -- (o.north);
  \draw[red,thick] (j.south) -- (s.north);
  \draw foreach \x in {l, m} {(\x.south) -- (q.north)};
  \draw[red,thick] foreach \x in {o, s} {(\x.south) -- (q.north)};
  \draw[black,->] (b) -- (c);
  \draw[black,->] (c) -- (d);
  \draw[black,->] (d) -- (e);
  \draw[black,->] (e) -- (f);
  \draw[black,->] (f) -- (g);
  \draw[black,->] (g) -- (r);
  \draw[black,->] (i) -- (j);
  \draw[black,->] (k) -- (l);
  \draw[black,->] (l) -- (m);
  \draw[black,->] (m) -- (n);
  \draw[black,->] (n) -- (o);
  \draw[black,->] (o) -- (p);
  \draw[black,->] (p) -- (s);
  \end{tikzpicture}
\end{exe}

The AR for the grammatical Finnish word \textipa{[p\o yt\ae-n\ae]}
`table (essive)', shown in (\ref{finnish.poytana}a), contains a single
-back and a single -neutral feature. Both features are associated to
each vowel in the word, which demonstrates full back and neutral
harmony. The hypothetical Finnish word, \textipa{[p\o yt\ae-nA]} in
(\ref{finnish.poytana}b) , however, contains the forbidden structure of
(\ref{fsc.finnish}) in bold and red because it demonstrates full neutral
harmony, but not full back harmony.

In (\ref{finnish.poytana}b) the suffix vowel does not harmonize with the
root-final vowel. While the Finnish pattern is captured by
(\ref{fsc.finnish}), the question remains whether including a morpheme
boundary in a representation would provide any necessary information.

\subsection{Turkish}\label{turkish}

Native Turkish words also demonstrate backness harmony where all suffix
vowels share the back feature of the root-final vowel, but it is debated
whether or not Turkish also utilizes back harmony within roots. The
vowel inventory of Turkish in Table \ref{turkish_vowels} consists of
eight vowels with three main featural distinctions: \(\pm\) high,
\(\pm\) back, \(\pm\) round. The current analysis will focus only on the
\(\pm\) high and \(\pm\) back features in order to analyze three-tiered
representations. In Turkish the +high -back vowels are {[}i, ü{]}, the
+high +back vowels are \textipa{[1, u]}, the -high -back vowels are
{[}e, ö{]}, and the -high +back vowels are {[}a, o{]}.

\begin{table}
  \caption{Turkish Vowels}
  \begin{tabular}{cc|c|cc|c|c|cc}
        & & \multicolumn{2}{|l}{-back}  &        & \multicolumn{2}{|l}{+back} &             & \\\cline{1-7}\cline{1-7}
  +high & & i                           & ü      &                            & \textipa{1} & u \\\cline{1-7}
  -high & & e                           & ö      &                            & a           & o \\\cline{1-7}\cline{1-7}
        & & -round                      & +round &                            & -round      & +round \\
  \end{tabular}
  \label{turkish_vowels}
\end{table}

The Turkish harmony generalization is that all suffix vowels share the
same back feature as the root-final vowel (Clements, 1976; Crothers \&
Shibatani, 1980; Nevins, 2010; Padgett, 2002; vanderHulst, 2017). For
example, the words in (\ref{turkish_back}) contain suffix vowels that
have the same back feature as the preceding root-final vowel regardless
of whether or not the high feature values are the same. A hyphen
represents a morpheme boundary.

\begin{exe}
\ex{suffix vowels share a back feature with root-final vowels}\label{turkish_back}
\begin{xlist}
  \ex \textipa{ip-ler} `rope (Nom.pl)'
  \ex \textipa{köy-ün} `village (Gen.sg)'
  \ex \textipa{el-i}   `hand (Acc.sg)'
  \ex \textipa{k1z-1n} `girl (Gen.sg)'
  \ex \textipa{son-u}  `end (Acc.sg)'
  \ex \textipa{pul-lar} `stamp (Nom.pl)'
\end{xlist}
\end{exe}

A straightforward pattern of backness harmony like Turkish can also be
written as an FSC that forbids vowels on either side of a morpheme
boundary from being associated to different back features, as in
(\ref{fsc.turkish}). No ordering relation is included on the back tier
because it forbids a root-final and suffix-initial vowel from being
associated to different back features regardless of which feature each
associates to. The melody tier also does not include an adjacency
ordering relation because the vowels can be separated by one or more
consonants and are necessarily separated by a morpheme boundary.

\begin{exe}
\ex \label{fsc.turkish}
  \begin{tikzpicture}[baseline=(current bounding box.north)]
  \matrix [matrix of nodes, row sep=2.5ex, column sep=2.25ex, nodes={text height=1em, text depth=0.5em}] 
  {
* & |(a)| +back &     & |(f)| -back \\
  & |(c)| V     & -- & |(h)| V \\
  };
  \draw (a.south) -- (c.north);
  \draw (f.south) -- (h.north);
  \end{tikzpicture}
\end{exe}

\begin{exe}
\ex{ip-ler `rope (Nom.pl)'}\label{turkish.ipler}
  \begin{tikzpicture}[baseline=(current bounding box.north)]
  \matrix [matrix of nodes, row sep=4ex, column sep=2.25ex, nodes={text height=1em, text depth=0.5em}] 
  {
(a) & |(a)| -back &              &         &         &         &         & (b) * & |[red](h)| \textbf{-back} &         &         &         & |[red](i)| \textbf{+back} \\ 
    & |(b)| i     & |(c)| p      & |(d)| -- & |(e)| l & |(f)| e & |(g)| r &       & |[red](j)| \textbf{i}                   & |(k)| p & |[red](l)| \textbf{--} & |(m)| l & |[red](n)| \textbf{a} & |(o)| r \\
  };
  \draw foreach \x in {b, f} {(a.south) -- (\x.north)};
  \draw[red,thick] (h.south) -- (j.north);
  \draw[red,thick] (i.south) -- (n.north);
  \draw[black,->] (b) -- (c);
  \draw[black,->] (c) -- (d);
  \draw[black,->] (d) -- (e);
  \draw[black,->] (e) -- (f);
  \draw[black,->] (f) -- (g);
  \draw[black,->] (j) -- (k);
  \draw[black,->] (k) -- (l);
  \draw[black,->] (l) -- (m);
  \draw[black,->] (m) -- (n);
  \draw[black,->] (n) -- (o);
  \end{tikzpicture}
\end{exe}

\newpage

\section{References}\label{references}

\setlength{\parindent}{-0.5in} \setlength{\leftskip}{0.5in}

\hypertarget{refs}{}
\hypertarget{ref-archangelipulleyblank1994}{}
Archangeli, D., \& Pulleyblank, D. (1994). \emph{Grounded phonology}
(Vol. 25). MIT Press.

\hypertarget{ref-Clements1976}{}
Clements, G. (1976). Vowel harmony in non-linear generative phonology:
An autosegmental model.

\hypertarget{ref-crothersshibatani1980}{}
Crothers, J., \& Shibatani, M. (1980). Issues in the description of
turkish vowel harmony. \emph{Issues in the Description of Turkish Vowel
Harmony}, 63--68.

\hypertarget{ref-Goldsmith1976}{}
Goldsmith, J. (1976). \emph{Autosegmental phonology} (PhD thesis).
Massachusetts Institute of Technology.

\hypertarget{ref-heinzidsardi2013}{}
Heinz, J., \& Idsardi, W. (2013). What complexity differences reveal
about domains in language. \emph{Topics in Cognitive Science},
\emph{5}(1), 111--131.

\hypertarget{ref-heinzetaltsl}{}
Heinz, J., Rawal, C., \& Tanner, H. G. (2011). Tier-based strictly local
constraints for phonology. In \emph{Proceedings of the 49th annual
meeting of the association for computational linguistics: Human language
technologies: Short papers} (Vol. 2). Association for Computational
Linguistics.

\hypertarget{ref-jardinediss}{}
Jardine, A. (2016). \emph{Locality and non-linear representations in
tonal phonology} (PhD thesis). University of Delaware.

\hypertarget{ref-jardinelocaltone}{}
Jardine, A. (2017). The local nature of tone association patterns.
\emph{Phonology}, \emph{34}(2), 385--405.

\hypertarget{ref-jardineexpressag}{}
Jardine, A. (2018). The expressivity of autosegmental grammars.

\hypertarget{ref-jardineheinz2015}{}
Jardine, A., \& Heinz, J. (2015). A concatenation operation to derive
autosegmental graphs. In \emph{Proceedings of the 14th annual meeting on
the mathematics of language (mol 2015)} (pp. 139--151). Chicago, USA:
Association for Computational Linguistics.

\hypertarget{ref-leben1973}{}
Leben, W. (1973). \emph{Suprasegmental phonology} (PhD thesis).
Massachusetts Institute of Technology.

\hypertarget{ref-mccarthyfg1988}{}
McCarthy, J. (1988). Feature geometry and dependency: A review.
\emph{Phonetica}, \emph{38}. Retrieved from
\url{http://scholarworks.umass.edu/linguist_faculty_pubs/38}

\hypertarget{ref-mccarthy2011}{}
McCarthy, J. (2011). Autosegmental spreading in optimality theory. In
\emph{Tones and features} (Clements Memorial Volume., Vol. 27).
Retrieved from
\url{https://scholarworks.umass.edu/linguist_faculty_pubs/27}

\hypertarget{ref-Nevins2010}{}
Nevins, A. (2010). \emph{Locality in vowel harmony}. \emph{Linguistic
Inquiry Monographs} (Vol. 55). MIT Press.

\hypertarget{ref-odden1994}{}
Odden, D. (1994). Adjacency parameters in phonology. \emph{Language},
\emph{70}(2), 289--330.

\hypertarget{ref-padgett1995}{}
Padgett, J. (1995). Feature classes. In J. Beckman, S. Urbanczyk, \& L.
Walsh (Eds.), \emph{Papers in optimality theory} (Vol. 18, pp.
385--420).

\hypertarget{ref-padgett2002}{}
Padgett, J. (2002). Feature classes in phonology. \emph{Language},
\emph{78}(1), 81--110. Retrieved from
\url{http://www.jstor.org/stable/3086646}

\hypertarget{ref-princesmolensky1993}{}
Prince, A., \& Smolensky, P. (1993). \emph{Optimality theory: Constraint
interaction in generative grammar} (No. 2). Rutgers University Center
for Cognitive Science.

\hypertarget{ref-princesmolensky2004}{}
Prince, A., \& Smolensky, P. (2004). \emph{Optimality theory: Constraint
interaction in generative grammar}. Blackwell.

\hypertarget{ref-ringenheinamaki1999}{}
Ringen, C., \& Heinamaki, O. (1999). Variation in finnish vowel harmony:
An ot account. \emph{Natural Languge and Linguistic Theory}, \emph{17},
303--337.

\hypertarget{ref-ringenvago1998}{}
Ringen, C., \& Vago, R. (1998). Hungarian vowel harmony in optimality.
\emph{Phonology}, \emph{15}, 393--416.

\hypertarget{ref-rogerspullum2011}{}
Rogers, J., \& Pullum, G. (2011). Aural pattern recognition experiments
and the subregular hierarchy. \emph{Journal of Logic, Language, and
Information}, \emph{20}, 329--342.

\hypertarget{ref-rogersetal2013}{}
Rogers, J., Heinz, J., Fero, M., Hurst, J., Lambert, D., \& Wibel, S.
(2013). Cognitive and sub-regular complexity. \emph{Formal Grammar},
90--108.

\hypertarget{ref-sagey1986}{}
Sagey, E. (1986). \emph{The representation of features and relations in
non-linear phonology} (PhD thesis). Massachusetts Institute of
Technology.

\hypertarget{ref-vdHulst2017}{}
vanderHulst, H. (2017). A representational account of vowel harmony in
terms of variable elements and licensing. In \emph{Approaches to
hungarian} (Vol. 15). John Benjamins Publishing Company.

\hypertarget{ref-valimaablum1986}{}
Välimaa-Blum, R. (1986). Finnish vowel harmony as a prescriptive and
descriptive rule: An autosegmental account. In F. Marshall (Ed.),
\emph{Proceedings of the third eastern states conference on
linguistics}. University of Pittsburgh.

\hypertarget{ref-walker2010}{}
Walker, R. (2010). Nonmyopic harmony and the nature of derivations.
\emph{Linguistic Inquiry}, \emph{41}(1), 169--179.

\hypertarget{ref-walkeroroquen}{}
Walker, R. (2014a). Nonlocal trigger-target relations. \emph{Linguistic
Inquiry}, \emph{45}(3), 501--523.

\hypertarget{ref-walker2014}{}
Walker, R. (2014b). Surface correspondence and discrete harmony
triggers. In \emph{Proceedings of the annual meetings on phonology}.






\end{document}
