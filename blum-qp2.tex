\documentclass[floatsintext,man]{apa6}

\usepackage{amssymb,amsmath}
\usepackage{ifxetex,ifluatex}
\usepackage{fixltx2e} % provides \textsubscript
\ifnum 0\ifxetex 1\fi\ifluatex 1\fi=0 % if pdftex
  \usepackage[T1]{fontenc}
  \usepackage[utf8]{inputenc}
\else % if luatex or xelatex
  \ifxetex
    \usepackage{mathspec}
    \usepackage{xltxtra,xunicode}
  \else
    \usepackage{fontspec}
  \fi
  \defaultfontfeatures{Mapping=tex-text,Scale=MatchLowercase}
  \newcommand{\euro}{€}
\fi
% use upquote if available, for straight quotes in verbatim environments
\IfFileExists{upquote.sty}{\usepackage{upquote}}{}
% use microtype if available
\IfFileExists{microtype.sty}{\usepackage{microtype}}{}

% Table formatting
\usepackage{longtable, booktabs}
\usepackage{lscape}
% \usepackage[counterclockwise]{rotating}   % Landscape page setup for large tables
\usepackage{multirow}		% Table styling
\usepackage{tabularx}		% Control Column width
\usepackage[flushleft]{threeparttable}	% Allows for three part tables with a specified notes section
\usepackage{threeparttablex}            % Lets threeparttable work with longtable

% Create new environments so endfloat can handle them
% \newenvironment{ltable}
%   {\begin{landscape}\begin{center}\begin{threeparttable}}
%   {\end{threeparttable}\end{center}\end{landscape}}

\newenvironment{lltable}
  {\begin{landscape}\begin{center}\begin{ThreePartTable}}
  {\end{ThreePartTable}\end{center}\end{landscape}}




% The following enables adjusting longtable caption width to table width
% Solution found at http://golatex.de/longtable-mit-caption-so-breit-wie-die-tabelle-t15767.html
\makeatletter
\newcommand\LastLTentrywidth{1em}
\newlength\longtablewidth
\setlength{\longtablewidth}{1in}
\newcommand\getlongtablewidth{%
 \begingroup
  \ifcsname LT@\roman{LT@tables}\endcsname
  \global\longtablewidth=0pt
  \renewcommand\LT@entry[2]{\global\advance\longtablewidth by ##2\relax\gdef\LastLTentrywidth{##2}}%
  \@nameuse{LT@\roman{LT@tables}}%
  \fi
\endgroup}


\ifxetex
  \usepackage[setpagesize=false, % page size defined by xetex
              unicode=false, % unicode breaks when used with xetex
              xetex]{hyperref}
\else
  \usepackage[unicode=true]{hyperref}
\fi
\hypersetup{breaklinks=true,
            pdfauthor={},
            pdftitle={On the locality of vowel harmony over multi-tiered autosegmental representations},
            colorlinks=true,
            citecolor=blue,
            urlcolor=blue,
            linkcolor=black,
            pdfborder={0 0 0}}
\urlstyle{same}  % don't use monospace font for urls

\setlength{\parindent}{0pt}
%\setlength{\parskip}{0pt plus 0pt minus 0pt}

\setlength{\emergencystretch}{3em}  % prevent overfull lines


% Manuscript styling
\captionsetup{font=singlespacing,justification=justified}
\usepackage{csquotes}
\usepackage{upgreek}



\usepackage{tikz} % Variable definition to generate author note

% fix for \tightlist problem in pandoc 1.14
\providecommand{\tightlist}{%
  \setlength{\itemsep}{0pt}\setlength{\parskip}{0pt}}

% Essential manuscript parts
  \title{On the locality of vowel harmony over multi-tiered autosegmental
representations}

  \shorttitle{Locality of vh over multi-tiered ARs}


  \author{Eileen Blum\textsuperscript{1}}

  % \def\affdep{{""}}%
  % \def\affcity{{""}}%

  \affiliation{
    \vspace{0.5cm}
          \textsuperscript{1} Rutgers University  }

  \authornote{
    Correspondence concerning this article should be addressed to Eileen
    Blum, 18 Seminary Place, New Brunswick, NJ 08901. E-mail:
    \href{mailto:eileen.blum@rutgers.edu}{\nolinkurl{eileen.blum@rutgers.edu}}
  }


  \keywords{keywords \\

    \indent Word count: X
  }




  \usepackage{tipa}
  \usepackage{gb4e}
  \noautomath
  \usepackage{tikz}
  \usetikzlibrary{matrix}
  \tikzset{marked/.style={draw=none, fill=none}}

\usepackage{amsthm}
\newtheorem{theorem}{Theorem}
\newtheorem{lemma}{Lemma}
\theoremstyle{definition}
\newtheorem{definition}{Definition}
\newtheorem{corollary}{Corollary}
\newtheorem{proposition}{Proposition}
\theoremstyle{definition}
\newtheorem{example}{Example}
\theoremstyle{definition}
\newtheorem{exercise}{Exercise}
\theoremstyle{remark}
\newtheorem*{remark}{Remark}
\newtheorem*{solution}{Solution}
\begin{document}

\maketitle

\setcounter{secnumdepth}{0}



\section{Finnish}\label{finnish}

Native Finnish words present a case of backness harmony with four
transparent front vowels. The Finnish vowel inventory, in Table
\ref{finnish_vowels}, consists of 16 vowels with contrastive length and
two main featural distinctions: \(\pm\) back and \(\pm\) neutral (Ringen
\& Heinamaki, 1999; Välimaa-Blum, 1986). The four vowels transparent to
backness harmony, \textipa{[i, i:, e, e:]}, are all +neutral and -back.
All other Finnish vowels are considered -neutral and distinguished by
backness such that the +back vowels are \textipa{[u, u:, o, o:, A, A:]}
and the -neutral, -back vowels are \textipa{[y, y:, \o, \o:, \ae, \ae:]}

\begin{table}
  \caption{Finnish Vowels}
  \begin{tabular}{c|c|c}
  \multicolumn{2}{l|}{-back}          & +back\\\hline\hline
  \textipa{i, i:} & \textipa{y, y:}   & \textipa{u, u:}\\ \cline{1-3}
  \textipa{e, e:} & \textipa{\o, \o:}  & \textipa{o, o:}\\ \cline{1-3}
                  & \textipa{\ae, \ae:} & \textipa{A, A:}\\ \cline{1-3}\hline\hline
  +neutral        & \multicolumn{2}{|l}{-neutral} \\
  \end{tabular}
  \label{finnish_vowels}
\end{table}

The harmony generalization is that all the -neutral vowels in a root
will share the same back feature with each other and the vowels in any
suffixes (Ringen \& Heinamaki, 1999; Välimaa-Blum, 1986). For example,
the words in (\ref{finnish_-neutral}) contain only -neutral vowels,
which are also all either +back or -back. A hyphen represents a morpheme
boundary.

\begin{exe}
\ex{-neutral vowels share a back feature value}\label{finnish_-neutral}
\begin{xlist}
  \ex \textipa{p\o yt\ae} `table' \\
  \ex \textipa{k\ae n-t\ae:} `turn' \\
  \ex \textipa{tyk\ae-t\ae} `like' \\
  \ex \textipa{poutA} `fine weather' \\
  \ex \textipa{mur-tA:} `break' \\
  \ex \textipa{kokA-tA} `cook' \\
  \end{xlist}
\end{exe}

The surface requirement that adjacent -neutral vowels share the same ATR
feature can also be written as a FSC, which forbids two adjacent vowels
associated to the same -neutral feature from being associated to
different back features, as in (\ref{fsc.finnish}), even across a
morpheme boundary. The ordering relation on the back tier in
(\ref{fsc.finnish}) is omitted because the + or - values of the two back
features are irrelevant for this constraint, as long as they differ. The
ordering relation on the melody tier of this FSC is also omitted and the
reason will be made clear by the example in (\ref{finnish.poytana}).

\begin{exe}
\ex \label{fsc.finnish}
  \begin{tikzpicture}[baseline=(current bounding box.north)]
  \matrix [matrix of nodes, row sep=2.5ex, column sep=2.25ex, nodes={text height=1em, text depth=0.5em}] 
  {
* & |(a)| +back & |(f)| -back \\
  & |(c)| V    & |(h)| V \\
  & |(e)| -neutral &  \\
  };
  \draw (a.south) -- (c.north);
  \draw (f.south) -- (h.north);
  \draw foreach \x in {c, h} {(\x.south) -- (e.north)};
  \end{tikzpicture}
\end{exe}

\begin{exe}
\ex{\textipa{[p\o yt\ae-n\ae]} `table (essive)'}\label{finnish.poytana}
  \begin{tikzpicture}[baseline=(current bounding box.north)]
  \matrix [matrix of nodes, row sep=4ex, column sep=2.25ex, nodes={text height=1em, text depth=0.5em}] 
  {
(a) &          & |(a)| -back    &         &         &           &         & \\ 
    & |(b)| p  & |(c)| \o       & |(d)| y & |(e)| t & |(f)| \ae & |(g)| n & |(r)| \ae \\
    &          & |(h)| -neutral &         &         &           &         &  \\          
(b) * &         & |[red](i)| \textbf{-back} &         &         &                                            &         & |[red](j)| \textbf{+back} \\
      & |(k)| p & |(l)| \o                  & |(m)| y & |(n)| t & |[red](o)| \textbf{\ae}                    & |(p)| n & |[red](s)| \textbf{\textipa{A}} \\
      &         & |[red](q)| \textbf{-neutral} &                       &         &                              &         & \\
  };
  \draw foreach \x in {c, d, f, r} {(a.south) -- (\x.north)};
  \draw foreach \x in {c, d, f, r} {(\x.south) -- (h.north)};
  \draw foreach \x in {l, m} {(i.south) -- (\x.north)};
  \draw[red,thick] (i.south) -- (o.north);
  \draw[red,thick] (j.south) -- (s.north);
  \draw foreach \x in {l, m} {(\x.south) -- (q.north)};
  \draw[red,thick] foreach \x in {o, s} {(\x.south) -- (q.north)};
  \draw[black,->] (b) -- (c);
  \draw[black,->] (c) -- (d);
  \draw[black,->] (d) -- (e);
  \draw[black,->] (e) -- (f);
  \draw[black,->] (f) -- (g);
  \draw[black,->] (g) -- (r);
  \draw[black,->] (i) -- (j);
  \draw[black,->] (k) -- (l);
  \draw[black,->] (l) -- (m);
  \draw[black,->] (m) -- (n);
  \draw[black,->] (n) -- (o);
  \draw[black,->] (o) -- (p);
  \draw[black,->] (p) -- (s);
  \end{tikzpicture}
\end{exe}

\newpage

\section{References}\label{references}

\setlength{\parindent}{-0.5in} \setlength{\leftskip}{0.5in}

\hypertarget{refs}{}
\hypertarget{ref-ringenheinamaki1999}{}
Ringen, C., \& Heinamaki, O. (1999). Variation in finnish vowel harmony:
An ot account. \emph{Natural Languge and Linguistic Theory}, \emph{17},
303--337.

\hypertarget{ref-valimaablum1986}{}
Välimaa-Blum, R. (1986). Finnish vowel harmony as a prescriptive and
descriptive rule: An autosegmental account. In F. Marshall (Ed.),
\emph{Proceedings of the third eastern states conference on
linguistics}. University of Pittsburgh.






\end{document}
